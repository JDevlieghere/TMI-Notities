\documentclass[examenvragen.tex]{subfiles}

\begin{document}
\section{Het algoritme van De Casteljau}
\subsection{Opgave}
Geef het algoritme van De Casteljau en bewijs de correctheid ervan.

\subsection{Antwoord}
Zij $b_{i}^{[0]} = b_{i}$ en definieer $b_{i}^{[k]}$ recursief als volgt.
\[
b_{i}^{[k]}
=
(1-t)b_{i-1}^{[k-1]} + tb_{i}^{[k-1]}
\]
Nu geldt volgende bewering.
\[
x = b_{n}^{[n]} = \sum_{i=0}^{n}b_iB_{i}^{n}(t)
\]
De B\'ezier curve kan dus recursief berekend worden.
\begin{proof}
\[
x(t) = \sum_{i=0}^{n}b_iB_{i}^{n}(t)
\]
Gebruik de recursiebetrekking voor bernstein-veeltermen.
\[
= \sum_{i=0}^{n}b_i
\left( 
(1-t)B^{n-1}_{i}(t) + tB^{n-1}_{i-1}(t)
\right)
= 
\sum_{i=0}^{n}b_i
(1-t)B^{n-1}_{i}(t) + 
\sum_{i=0}^{n}b_it
B^{n-1}_{i-1}(t)
\]
De eerste en laatste term van deze uitdrukking kunnen we weglaten ($B_{i}^{n} = 0$ als $i < 0$ of $i > n$).
\[
= 
\sum_{i=0}^{n-1}b_i
(1-t)B^{n-1}_{i}(t) + 
\sum_{i=1}^{n}b_it
B^{n-1}_{i-1}(t)
\]
Merk op dat alle $b_i$ gelijk zijn aan $b_i^{[0]}$. We proberen ze nu zo te schrijven dat we er $b_i^{[1]}$ van kunnen maken. Vervang de $i$ in de eerste term door $j=i+1$ en hernoem daarna $j$ naar $i$. In feite schuiven we de sommatie-index $1$ \'e\'enheid op.
\[
= 
\sum_{i=1}^{n}b_{i-1}
(1-t)B^{n-1}_{i-1}(t) + 
\sum_{i=1}^{n}b_it
B^{n-1}_{i-1}(t)
\]
Nu zien we dat de sommaties dezelfde zijn. We kunnen deze dus buiten brengen omdat de optelling commutatief is.
\[
= 
\sum_{i=1}^{n}
\left(
b_{i-1}(1-t)B^{n-1}_{i-1}(t)
+ 
b_itB^{n-1}_{i-1}(t)
\right)
\]
\[
= 
\sum_{i=1}^{n}
B^{n-1}_{i-1}(t)
\left(
(1-t)b_{i-1}
+ 
tb_i
\right)
\]
Kijk nu naar de zojuist gedefini\"eerde recursiebetrekking. We kunnen $(1-t)b_{i-1} + tb_i = (1-t)b_{i-1}^{[0]} + tb_i^{[0]}$ vervangen door $b_{i}^{[1]}$
\[
= 
\sum_{i=1}^{n}
B^{n-1}_{i-1}(t)
b_{i}^{[1]}
\]
Als we dit proces nog eens herhalen krijgen we de volgende uitdrukking.
\[
= 
\sum_{i=2}^{n}
B^{n-2}_{i-2}(t)
b_{i}^{[2]}
\]
In de $j$-de iteratie ziet het nieuwe rechterlid er als volgt uit.
\[
= 
\sum_{i=j}^{n}
B^{n-j}_{i-j}(t)
b_{i}^{[j]}
\]
Dit proces kunnen we $n$ (eindig en zelfs een lineair aantal) keer uitvoeren tot we volgende uitdrukking bekomen.
\[
\sum_{i=n}^{n}
B^{0}_{0}(t)
b_{i}^{[n]}
= b_{n}^{[n]}
\]
\end{proof}
Deze bewering geeft een algoritme om een punt op de B\'ezier-curve recursief te berekenen. De index in superscript tussen vierkante haakjes heeft betrekking tot de iteratie waarin dit punt zich bevindt. In de praktijk wordt het principe van ``dynamic'' programming ingezet, zodat het algoritme een complexiteit heeft van $O(n^2)$ in plaats van $O(2^n)$.

\end{document}
