\documentclass[12pt,a4paper]{article}
\usepackage[latin1]{inputenc}

\usepackage[dutch]{babel}
\usepackage{listings}

% Voor algoritmes
\usepackage{algorithm2e}

\usepackage{listings}
\lstset{ %
  language=Java,               	  % the language of the code
  basicstyle=\footnotesize,       % the size of the fonts that are used for the code
  numbers=left,                   % where to put the line-numbers
  stepnumber=1,                   % the step between two line-numbers. If it's 1, each line
                                  % will be numbered
  numbersep=5pt,                  % how far the line-numbers are from the code
  showspaces=false,               % show spaces adding particular underscores
  showstringspaces=false,         % underline spaces within strings
  showtabs=false,                 % show tabs within strings adding particular underscores
  frame=single,                   % adds a frame around the code
  tabsize=2,                      % sets default tabsize to 2 spaces
  captionpos=b,                   % sets the caption-position to bottom
  breaklines=true,                % sets automatic line breaking
  breakatwhitespace=false,        % sets if automatic breaks should only happen at whitespace
  title=\lstname,                 % show the filename of files included with \lstinputlisting;
}

% Voor todo's
\usepackage{todonotes}

% Voor wiskunde
\usepackage{amsmath}
\usepackage{amsfonts}
\usepackage{amssymb}
\usepackage{amsthm}

% Voor urls
\usepackage{hyperref}

% svg
\usepackage[clean,pdf]{svg}
\setsvg{svgpath = illustraties/}
\setsvg{inkscape = inkscape -z -D}


% Om het totaal aantal pagina's te tellen
\usepackage{lastpage}
\usepackage{afterpage}

% Voor tekeningen
\usepackage{tikz}
\usetikzlibrary{decorations}
\usetikzlibrary{calc}
\usetikzlibrary{arrows.meta}

% Nog tekeningen
\usepackage{pgfplots}

% SVG tekeningen
\usepackage{svg}

% Om figuren op de juiste plaats te krijgen
\usepackage{float}

% Om de marges aan te passen
\usepackage[left=2cm,right=2cm,top=2cm,bottom=2cm]{geometry}


\usepackage{titlesec}
\usepackage{subfiles}
\usepackage{multicol}
\usepackage{wrapfig}
\usepackage{pdfpages}

% Voor headers en footers
\usepackage{fancyhdr}
% fancy verbatim
\usepackage{fancyvrb}
% program listings
\usepackage{listings}
% clickable TOC
\usepackage{hyperref}
\hypersetup{
    colorlinks,
    citecolor=black,
    filecolor=black,
    linkcolor=black,
    urlcolor=black
}

\author{Tom Sydney Kerckhove}
\title{Examenvragen TMI}

\usepackage{titlesec}

\renewcommand\thesection{Vraag \arabic{section}}
\renewcommand\thesubsection{V \arabic{section}}

\begin{document}

\pagebreak

\begin{titlepage}
\thispagestyle{empty}
\newcommand{\HRule}{\rule{\linewidth}{0.5mm}}
\center
\textsc{\LARGE KU Leuven}\\[1.5cm]
\vfill


{ \Huge \bfseries Computergestuurd ontwerp van curven en vlakken}\\[0.4cm]
% \HRule \\[1.5cm]
\textsc{\large Toepassingen van de meetkunde in de informatica [G0Q37C]}\\[0.5cm]

\vspace{5cm}

\begin{Large}
Gestart: 21 februari 2014\\
Cecompileerd: \today\\
\end{Large}
\vspace{5cm}

\begin{minipage}{0.4\textwidth}
\begin{flushleft} \large
\emph{Auteur:}\\
Tom Sydney \textsc{Kerckhove}
\end{flushleft}
\end{minipage}
~
\begin{minipage}{0.4\textwidth}
\begin{flushright} \large
\emph{Professor:} \\
prof. dr. ir. Dirk \textsc{Roose}\\
\end{flushright}
\end{minipage}\\[4cm]

\vfill

\end{titlepage}



\part{B\'ezier-curven}
\subfile{vraag_c2_continuiteit}
\subfile{vraag_de_casteljau}
\subfile{vraag_variatieverminderingseigenschap}
\subfile{vraag_graadverhoging}
\subfile{vraag_subdivisie}

\iffalse
\section{Bespreek tensorproductoppervlakken aan de hand van B\'ezier-oppervlakken.}
\section{Bereken de tweede partiele afgeleide in de hoekpunten van een B\'ezier-oppervlak, leg uit en geef grafisch weer.}
\fi

\part{Spline-curven}
\subfile{vraag_eenheidspartitie_bsplines}
\subfile{vraag_splinecurve_interpolatie}
\subfile{vraag_de_boor}
\subfile{vraag_recht_lijnstuk_splinecurve}


\part{Algemeen}
\subfile{vraag_bounding_box_test}
\subfile{vraag_bruggen}
\subfile{vraag_verste_puntenpaar}

\iffalse
\section{bespreek graham scan + correctheidsbewijs + toon aan dat dit O(nlogn)bewerkingen gebeurt}
\section{wanneer is de inpakmethode (jarvis march) efficienter dan de methode van graham (graham scan)}
\fi

\part{Voronoi diagrammen}
\subfile{vraag_voronoi_punten_en_zijden}
\subfile{vraag_maximale_lege_cirkel}
\subfile{vraag_fortune_event_points}
\iffalse
\section{Bewijs: ``Een voronoi veelhoek van een punt $p_i$ is begrensd <=> $p_i$ element van $inw(CH(S))$'' Bespreek het nut van deze stelling.}
\section{Bewijs dat minimale doorloopboom deelverzameling is van de Delaunay triangulatie. Wat is het nut van deze eigenschap?}
\section{Bewijs: "Twee dichtste buren hebben een gemeenschappelijke voronoi zijde"}


\fi

\part{Nabijheidsproblemen}
\subfile{vraag_transformeerbaarheid}
\subfile{vraag_emdb}

\part{Padplanning}
\subfile{vraag_robotarm}

\part{Oefeningen}
\subfile{vraag_diagonaal}
\subfile{vraag_grootste_oppervlakte}
\subfile{vraag_overlappende_veelhoeken}
\subfile{vraag_overlappende_rechthoeken}
\subfile{vraag_snijpunten_lijnstukken}
\subfile{vraag_zichtbare_lijnstukken}
\subfile{vraag_snijpunten_van_cirkels}
\subfile{vraag_alle_snijpunten_van_cirkels}
\subfile{vraag_punten_in_cirkels}
\subfile{vraag_minowski_som}
\subfile{vraag_drie_link_probleem}



\end{document}