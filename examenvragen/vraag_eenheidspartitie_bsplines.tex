\documentclass[examenvragen.tex]{subfiles}

\begin{document}

\section{Eenheidspartitie splines}
\subsection{Opgave}
Bewijs de eenheidspartitie eigenschap van genormaliseerde B-splines.
\[
\forall u \in [u_0,u_p)\ :\
\sum_{i=-\infty}^{\infty}N_{i}^{n}(u)
= \sum_{i=-n}^{p-1}N_{i}^{n}(u)
= 1
\]

\subsection{Antwoord}
\begin{proof}Bewijs door inductie.\\
\begin{enumerate}
\item De bewering geldt voor $n=0$. (Basis)\\
Inderdaad, voor elk interval $u_j,u_{j+1}$ is er precies \'e\'en $N_{i}^{0}$ die gelijk is aan \'e\'en terwijl de andere nul zijn.
\[
u \in [u_0,u_p) \Rightarrow
\sum_{i=0}^{p-1}N_{i}^{0}(u)
= 1
\]
\item De bewering geldt voor $n=k$. (Inductiehyposhese)
\[
\forall u \in [u_0,u_p)\ :\
\sum_{i=-k}^{p-1}N_{i}^{k}(u)
= 1
\]
\item De bewering geldt voor $n=k+1$. (Inductiestap)\\
We bewijzen nu dat als de bewering geldt voor een bepaalde $n=k$, de bewering dan ook geldt voor $n=k+1$.
We bewijzen dus dat uit de inductiehypothese volgt dat de volgende bewering geldt.
\[
\forall u \in [u_0,u_p) \Rightarrow
\sum_{i=-k-1}^{p-1}N_{i}^{k+1}(u)
= 1
\]
Schrijf het rechter lid uit met behulp van de recursiebetrekking voor B-splines.
\[
\sum_{i=-k-1}^{p-1}N_{i}^{k+1}(u)
= \sum_{i=-k-1}^{p-1}
\left(
\frac{u-u_i}{u_{i+k+1}-u_i}					N_{i}^{k}(u)
+ \frac{u_{i+k+2}-u}{u_{i+k+2}-u_{i+1}}			N_{i+1}^{k}(u)
\right)
\]
Splits de sommatie
\[
=
\sum_{i=-k-1}^{p-1}
\frac{u-u_i}{u_{i+k+1}-u_i}					N_{i}^{k}(u)
+ 
\sum_{i=-k-1}^{p-1}
\frac{u_{i+k+2}-u}{u_{i+k+2}-u_{i+1}}			N_{i+1}^{k}(u)
\]
In deze uitdrukking zijn de eerste en de laatste term beide nul
\[
=
\sum_{i=-k}^{p-1}
\frac{u-u_i}{u_{i+k+1}-u_i}					N_{i}^{k}(u)
+ 
\sum_{i=-k-1}^{p-2}
\frac{u_{i+k+2}-u}{u_{i+k+2}-u_{i+1}}			N_{i+1}^{k}(u)
\]
De tweede som kan herschreven worden door de index aan te passen als volgt.
\[
\sum_{i=-k-1}^{p-2}
\frac{u_{i+k+2}-u}{u_{i+k+2}-u_{i+1}}			N_{i+1}^{k}(u)
=
\sum_{i=-k}^{p-1}
\frac{u_{i+k+1}-u}{u_{i+k+1}-u_{i}}			N_{i}^{k}(u)
\]
We vervangen dit nu zodat we verder gaan met de volgende uitdrukking.
\[
=
\sum_{i=-k}^{p-1}
\frac{u-u_i}{u_{i+k+1}-u_i}					N_{i}^{k}(u)
+ 
\sum_{i=-k}^{p-1}
\frac{u_{i+k+1}-u}{u_{i+k+1}-u_{i}}			N_{i}^{k}(u)
\]
\[
=
\sum_{i=-k}^{p-1}
\left(
\frac{u-u_i}{u_{i+k+1}-u_i}					N_{i}^{k}(u)
+ 
\frac{u_{i+k+1}-u}{u_{i+k+1}-u_{i}}			N_{i}^{k}(u)
\right)
\]
\[
=
\sum_{i=-k}^{p-1}
\frac{u-u_i + u_{i+k+1}-u}{u_{i+k+1}-u_{i}} N_{i}^{k}(u)
=
\sum_{i=-k}^{p-1}
N_{i}^{k}(u)
\]
Uit de inductiehypothese volgt precies dat deze uitdrukking gelijk is aan \'e\'en. Hiermee is de stelling bewezen.
\end{enumerate}
\end{proof}
\end{document}
