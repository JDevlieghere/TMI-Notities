\documentclass[examenvragen.tex]{subfiles}

\begin{document}


\section{Snijpunten van cirkels}
\subsection{Generisch}
\subsubsection{Opgave}
Gegeven een convexe veelhoek $A$ (de robot) en een niet-convexe, maar wel eenvoudige veelhoek $B$ (het obstakel). Bereken de regio's waar de robot niet kan komen met behulp van een Minowki som.

\subsubsection{Antwoord}
Wanneer we de robot vervangen door een puntrobot kunnen we met behulp van een Minowski som de onbereikbare gebieden berekenen. Spiegel de robot ten opzichte van het gekozen referentiepunt om $A'$ te bekomen. Bereken de Minowski som van $A'$ met $B$ met behulp van het sterdiagram.

Benoem de zijden van de robot veelhoek met letters en de zijden van het obstakel met getallen.

Om het sterdiagram te construeren beginnen we met een punt dat we het sterpunt noemen. Een gerichte zijde toevoegen aan de sterdiagram houdt in dat we de zijde kopi\"eren zodat het beginpunt overeenkomt met het sterpunt. Ga beide veelhoeken af, tegen de klok in, zijde na zijde, en voeg elke zijde toe aan het sterdiagram.

Om de som te berekenen gaan we het sterdiagram af in tegenwijzerzin. Teken elke zijde die je tegenkomt achtereenvolgens, maar sla de zijde over als je twee getallen na elkaar tegen komt.



\end{document}
