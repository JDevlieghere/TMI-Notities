\documentclass[examenvragen.tex]{subfiles}

\begin{document}
\section{Graham scan}
\subsection{Opgave}
Bespreek het Graham scan algoritme.

\subsection{Antwoord}
Het Graham scan algoritme voor het berekenen van de convex omhullende van een verzameling punten werkt als volgt.
\begin{enumerate}
\item Sorteer de punten volgens stijgende poolhoek.
\item Ga de punten \'e\'en voor \'e\'en af, en voeg het punt enkel toe als het geen rechter draai maakt ten opzichte van het vorige punt.
\item Wanneer alle punten overlopen zijn, is de gehele convex omhullende gevonden.
\end{enumerate}
Het Graham scan algoritme vindt enkel punten op de convex omhullende en vindt ze allemaal. Het sorteren van de punten volgens poolhoek kost $O(N\log(N))$ tijd. Daarna worden alle punten \'e\'en voor \'e\'en overlopen in $O(1)$ tijd.



\end{document}
