\documentclass[computergesteund_ontwerp_van_curven_en_oppervlakken.tex]{subfiles}
\begin{document}

\chapter{Splinecurven}

\section{Knooppunten en `de Boor-punten'}
\begin{de}
\label{knooppunten}
We beschouwen een parameter-interval $[u_0,u_p]$ met $p+1$ \textbf{knooppunten} $u_0<u_1<\cdots<u_p$.
\end{de}
\begin{de}
\label{deboorpunten}
Controlepunten voor een splinecurve noemt men \textbf{`de Boor-punten'}.
\end{de}
\section{B-Splines}
\begin{de}
\label{bspline}
Een \textbf{B-spline} functie is een basis functie. Het is een stuksgewijze veelterm die enkel lokaal niet nul is. Een B-spline heeft een index $i$ die bij een knooppunt $u_i$ hoort en een graad $n$.
\[
N_{i}^{n}(u)
= \frac{u-u_i}{u_{i+n}-u_i}					N_{i}^{n-1}(u)
+ \frac{u_{i+n+1}-u}{u_{i+n+1}-u_{i+1}}		N_{i+1}^{n-1}(u)
\]
met
\[
u \in [u_i,u_{i+1}) \Rightarrow N_{i}^{0}(u) = 1
\]
\[
u \not\in [u_i,u_{i+1}) \Rightarrow N_{i}^{0}(u) = 0
\]
\end{de}

\begin{ei}\textbf{Lokaliteit}
\label{bspline_lokaliteit}
\[
u \not\in [u_i,u_{i+n+1}] \Rightarrow N_{i}^{n}(u) = 0
\]
Een B-spline is nul buiten haar lokaal interval $[u_i,u_{i+n+1}]$.

\begin{proof} door inductie.\\
\begin{enumerate}
\item De bewering geldt voor $n=0$. (Basis)
\[
u \in [u_i,u_{i+1}) \Rightarrow N_{i}^{0}(u) = 1
\]
\[
u \not\in [u_i,u_{i+1}) \Rightarrow N_{i}^{0}(u) = 0
\]
\item De bewering geldt voor $n=k$. (Inductiehyposhese)
\[
u \not\in [u_i,u_{i+k+1}] \Rightarrow N_{i}^{k}(u) = 0
\]
\item De bewering geldt voor $n=k+1$. (Inductiestap)\\
We bewijzen nu dat als de bewering geldt voor een bepaalde $n=k$, de bewering dan ook geldt voor $n=k+1$.
Te bewijzen is nu het volgende.
\[
u \not\in [u_i,u_{i+k+2}] \Rightarrow N_{i}^{k+1}(u) = 0
\]
We weten het volgende over $N_{i}^{k+1}(u)$.
\[
N_{i}^{k+1}(u)
= \frac{u-u_i}{u_{i+k+1}-u_i}					N_{i}^{k}(u)
+ \frac{u_{i+k+2}-u}{u_{i+k+2}-u_{i+1}}			N_{i+1}^{k}(u)
\]
Deze uitdrukking is nul wanneer $N_{i}^{k}(u)$ en $N_{i+1}^{k}(u)$ beide nul zijn. Inderdaad, buiten $[u_i,u_{i+k+1}]$ is $N_{i}^{k}(u)$ nul en buiten $[u_{i+1},u_{i+k+2}]$ is $N_{i+1}^{k}(u)$ nul.
\[
[u_i,u_{i+k+1}] \subset [u_i,u_{i+k+2}]
\text{ en }
[u_{i+1},u_{i+k+2}] \subset [u_i,u_{i+k+2}]
\]
\end{enumerate}
\end{proof}
\end{ei}

\begin{ei}\textbf{Positiviteit}
\label{bspline_positiviteit}
\[
N_{i}^{n}(u) \ge 0
\]
\begin{proof} door inductie.\\
\begin{enumerate}
\item De bewering geldt voor $n=0$. (Basis)\\
Inderdaad, $N_{i}^{0}$ kan enkel $0$ of $1$ zijn.
\item De bewering geldt voor $n=k$. (Inductiehyposhese)
\[
N_{i}^{k}(u) \ge 0
\]
\item De bewering geldt voor $n=k+1$. (Inductiestap)\\
We bewijzen nu dat als de bewering geldt voor een bepaalde $n=k$, de bewering dan ook geldt voor $n=k+1$.
We bekijken de uitdrukking voor $N_{i}^{k+1}(u)$.
\[
N_{i}^{k+1}(u)
= \frac{u-u_i}{u_{i+k+1}-u_i}					N_{i}^{k}(u)
+ \frac{u_{i+k+2}-u}{u_{i+k+2}-u_{i+1}}			N_{i+1}^{k}(u)
\]
Deze uitdrukking zou enkel negatief kunnen zijn als $u < u_i$ of $u_{i+n+1} < u$ geldt. (uit de inductiehypothese volgt dat $N_{i}^{k}(u)$ en $N_{i+1}^{k}(u)$ positief zijn.).
$N_{i}^{k+1}(u)$ is echter steeds nul wanneer $u \not\in [u_i,u_{i+k+2}]$\footnote{Zie eigenschap \ref{bspline_positiviteit} van B-splines.} en binnen $[u_i,u_{i+k+2}]$ is $u$ nooit kleiner dan $u_i$ of groter dan $u_{i+k+2}$.
\end{enumerate}
\end{proof}
\end{ei}

\begin{ei}
\textbf{Eenheidspartitie} (Sommatie tot $1$)
\label{bspline_eenheidspartitie}
\[
\forall u \in [u_0,u_p)\ :\
\sum_{i=-\infty}^{\infty}N_{i}^{n}(u)
= \sum_{i=-n}^{p-1}N_{i}^{n}(u)
= 1
\]

\begin{proof}Bewijs door inductie.\\
\begin{enumerate}
\item De bewering geldt voor $n=0$. (Basis)\\
Inderdaad, voor elk interval $u_j,u_{j+1}$ is er precies \'e\'en $N_{i}^{0}$ die gelijk is aan \'e\'en terwijl de andere nul zijn.
\[
u \in [u_0,u_p) \Rightarrow
\sum_{i=0}^{p-1}N_{i}^{0}(u)
= 1
\]
\item De bewering geldt voor $n=k$. (Inductiehyposhese)
\[
\forall u \in [u_0,u_p)\ :\
\sum_{i=-k}^{p-1}N_{i}^{k}(u)
= 1
\]
\item De bewering geldt voor $n=k+1$. (Inductiestap)\\
We bewijzen nu dat als de bewering geldt voor een bepaalde $n=k$, de bewering dan ook geldt voor $n=k+1$.
We bewijzen dus dat uit de inductiehypothese volgt dat de volgende bewering geldt.
\[
\forall u \in [u_0,u_p) \Rightarrow
\sum_{i=-k-1}^{p-1}N_{i}^{k+1}(u)
= 1
\]
Schrijf het rechter lid uit met behulp van de recursiebetrekking voor B-splines\footnote{Zie definitie \ref{bspline}.}.
\[
\sum_{i=-k-1}^{p-1}N_{i}^{k+1}(u)
= \sum_{i=-k-1}^{p-1}
\left(
\frac{u-u_i}{u_{i+k+1}-u_i}					N_{i}^{k}(u)
+ \frac{u_{i+k+2}-u}{u_{i+k+2}-u_{i+1}}			N_{i+1}^{k}(u)
\right)
\]
Splits de sommatie
\[
=
\sum_{i=-k-1}^{p-1}
\frac{u-u_i}{u_{i+k+1}-u_i}					N_{i}^{k}(u)
+ 
\sum_{i=-k-1}^{p-1}
\frac{u_{i+k+2}-u}{u_{i+k+2}-u_{i+1}}			N_{i+1}^{k}(u)
\]
In deze uitdrukking zijn de eerste en de laatste term beide nul.\footnote{Zie eigenschap \ref{bspline_lokaliteit} van B-splines.}
\[
=
\sum_{i=-k}^{p-1}
\frac{u-u_i}{u_{i+k+1}-u_i}					N_{i}^{k}(u)
+ 
\sum_{i=-k-1}^{p-2}
\frac{u_{i+k+2}-u}{u_{i+k+2}-u_{i+1}}			N_{i+1}^{k}(u)
\]
De tweede som kan herschreven worden door de index aan te passen als volgt.
\[
\sum_{i=-k-1}^{p-2}
\frac{u_{i+k+2}-u}{u_{i+k+2}-u_{i+1}}			N_{i+1}^{k}(u)
=
\sum_{i=-k}^{p-1}
\frac{u_{i+k+1}-u}{u_{i+k+1}-u_{i}}			N_{i}^{k}(u)
\]
We vervangen dit nu zodat we verder gaan met de volgende uitdrukking.
\[
=
\sum_{i=-k}^{p-1}
\frac{u-u_i}{u_{i+k+1}-u_i}					N_{i}^{k}(u)
+ 
\sum_{i=-k}^{p-1}
\frac{u_{i+k+1}-u}{u_{i+k+1}-u_{i}}			N_{i}^{k}(u)
\]
\[
=
\sum_{i=-k}^{p-1}
\left(
\frac{u-u_i}{u_{i+k+1}-u_i}					N_{i}^{k}(u)
+ 
\frac{u_{i+k+1}-u}{u_{i+k+1}-u_{i}}			N_{i}^{k}(u)
\right)
\]
\[
=
\sum_{i=-k}^{p-1}
\frac{u-u_i + u_{i+k+1}-u}{u_{i+k+1}-u_{i}} N_{i}^{k}(u)
=
\sum_{i=-k}^{p-1}
N_{i}^{k}(u)
\]
Uit de inductiehypothese volgt precies dat deze uitdrukking gelijk is aan \'e\'en. Hiermee is de stelling bewezen.
\end{enumerate}
\end{proof}
\end{ei}

\begin{ei}
\textbf{Continu\"iteit}\\
$N_{i}^{n}(u)$ is een $C_{n-1}$ continue functie indien er geen meervoudige knooppunten voorkomen. \footnote{De continu\"iteitseigenschap kan niet bewezen worden via de recursiebetrekking. Wanneer we B-splines op een andere manier (equivalent) defini\"eren valt dit wel te bewijzen.}
\end{ei}

\section{Splinecurve}
\begin{de}
\[
s(u) = \sum_{i=-n}^{p-1}d_iN_{i}^{n}(u)
\]
Bij een $n$-de graads \textbf{B-splinecurve} met $p+1$ knooppunten $u_i$ met $i = 0,\cdots ,p$ horen dus $p+n$ `de Boor-punten' $d_i$ met $i = -n, -n+1, \cdots, 0, \cdots , p-1$. Er zijn dan nog $2n$ extra knooppunten nodig: $u_{-n},\cdots,u_{-1}$ en $u_{p+1},\cdots,u_{p+n}$.
\end{de}


\begin{ei}
Spline curves zijn invariant onder affiene transformaties.
\begin{proof}
Een punt op een B-spline curve is een affiene (zelfs convexe) combinatie van de `de Boor-punten'. \footnote{Zie eigenschap \ref{bspline_eenheidspartitie} (en eigenschap \ref{bspline_positiviteit}) van B-splines.}
Een affiene combinatie is invariant onder affiene transformaties.\footnote{Zie definitie \ref{affiene_transformatie}.}
\end{proof}
\end{ei}

\begin{ei}
Bij wijziging van \'e\'en `de Boor' punt $d_i$ zal niet heel de curve veranderen. Enkel het deel dat overeenstemt met $u \in [u_i,u_{i+n+1})$.
\begin{proof}
B-splines buiten het interval $[u_i,u_{i+n+1})$ zijn steeds nul.\footnote{Zie eigenschap \ref{bspline_lokaliteit} van B-splines.}
De invloed van een verandering in $d_i$ in buiten $[u_i,u_{i+n+1})$ is dus onbestaande.
\end{proof}
\end{ei}

\begin{ei}
Spline curves liggen binnen de convex omhullende van de `de Boor' punten. Sterker nog, de curve ligt steeds binnen de convex omhullende van de $n+1$ omliggende `de Boor punten'.

\begin{proof}
In het deel van de curve dat overeenstemt met $u \in [u_j,u_{j+1})$ zijn enkel de B-splines $N_{j-n}^{n},..., N_{j}^{n}$ niet nul.\footnote{Zie eigenschap \ref{bspline_lokaliteit} van B-splines.} Een punt op de B-splinecurve is een affiene\footnote{Zie eigenschap \ref{bspline_eenheidspartitie} van B-splines.} en zelfs convexe\footnote{Zie eigenschap \ref{bspline_positiviteit} van B-splines.} combinatie van $n+1$ `de Boor punten'. Van een convexe combinatie weten we bovendien dat ze binnen de convex omhullende ligt.\footnote{Zie eigenschap \ref{combinatie_in_omhullende} van convexe combinaties.}
\end{proof}
\end{ei}




\section{Evaluatie van een punt op een B-Spline curve (de Boor)}
Gegeven een curve $s(u)$ gedefinie\"erd over een interval $u_0,u_p$. 
\[
s(u) = \sum_{i=-n}^{p-1}d_iN_{i}^{n}(u)
\]
Gevraagd is het punt op $s$ dat overeen komt met $u\in [u_l,u_{l+1})$. Het punt $s(u)$ hangt af van $n$ `de Boor punten' $d_{l-n},\cdots,d_{l}$. De andere `de Boor punten' spelen geen rol.\footnote{Zie eigenschap \ref{bspline_lokaliteit} van B-splines.}
\[
s(u) = \sum_{i=l-n}^{l}d_{i}N_{i}^{n}(u)
\]
\begin{st}
$s(u)$ is $d_{l}^{[k]}$ waarbij $d_{l}^{[k]}$ als volgt gedefinieerd is.
\[
d_{i}^{[k]}
= (1-\alpha_{k,i})d_{i-1}^{[k-1]}
+ \alpha_{k,i}d_{i}^{[k-1]}
\text{ met }
\alpha_{k,i} = \frac{u-u_i}{u_{i+n+1-k}-u_{i}}
\]

\begin{proof}
\[
s(u)
= \sum_{i=-n}^{p-1}d_iN_{i}^{n}(u)
\underset{u \in [u_l,u_{l+1})}{=}
\sum_{i=l-n}^{l}d_iN_{i}^{n}(u)
\]
\[
=
\sum_{i=l-n}^{l}d_i
\left(
\frac{u-u_i}{u_{i+n}-u_i}N_i^{n-1}(u) + 
\frac{u_{i+n+1}-u}{u_{i+n+1}-u_{i+1}}N_{i+1}^{n-1}(u)
\right)
\]
Splits de sommatie.
\[
=
\sum_{i=l-n}^{l}d_i
\frac{u-u_i}{u_{i+n}-u_i}N_i^{n-1}(u) + 
\sum_{i=l-n}^{l}d_i
\frac{u_{i+n+1}-u}{u_{i+n+1}-u_{i+1}}N_{i+1}^{n-1}(u)
\]
De eerste en de laatste term in deze zijn nul.\footnote{Zie eigenschap \ref{bspline_lokaliteit} van B-splines.}
\[
=
\sum_{i=l-n+1}^{l}d_i
\frac{u-u_i}{u_{i+n}-u_i}N_i^{n-1}(u) + 
\sum_{i=l-n}^{l-1}d_i
\frac{u_{i+n+1}-u}{u_{i+n+1}-u_{i+1}}N_{i+1}^{n-1}(u)
\]
De tweede som in bovenstaande uitdrukking valt te herschrijven als volgt.
\[
\sum_{i=l-n}^{l-1}d_i
\frac{u_{i+n+1}-u}{u_{i+n+1}-u_{i+1}}N_{i+1}^{n-1}(u)
=
\sum_{i=l-n+1}^{l}d_{i-1}
\frac{u_{i+n}-u}{u_{i+n}-u_{i}}N_{i}^{n-1}(u)
\]
We kunnen dit dus vervangen in de originele vergelijking om verder te gaan.
\[
=
\sum_{i=l-n+1}^{l}d_i
\frac{u-u_i}{u_{i+n}-u_i}N_i^{n-1}(u) + 
\sum_{i=l-n+1}^{l}d_{i-1}
\frac{u_{i+n}-u}{u_{i+n}-u_{i}}N_{i}^{n-1}(u)
\]
\[
s(u)
=
\sum_{i=l-n+1}^{l}
N_i^{n-1}(u)
\left(
d_{i-1}
\frac{u-u_i}{u_{i+n}-u_i} +
d_i
\frac{u_{i+n}-u}{u_{i+n}-u_{i}}
\right)
\]
\[
s(u)
= \sum_{i=l-n+1}^{l}
d_{i}^{[1]}
N_i^{n-1}(u)
\]
Bovenstaan proces kunnen we herhalen om de volgende uitdrukking te krijgen.
\[
s(u)
= \sum_{i=l-n+2}^{l}
d_{i}^{[2]}
N_i^{n-2}(u)
\]
Na $j$ iteraties bekomen we dan volgende uitdrukking ($j\le n$)
\[
s(u)
= \sum_{i=l-n+j}^{l}
d_{i}^{[j]}
N_i^{n-j}(u)
\]
Na $n$ iteraties bekomen we dan hetgeen te bewijzen viel.
\[
s(u) = \sum_{i=l}^{l}d_{i}^{[n]}N_{i}^{0}(u) = d_{i}^{[n]}
\]
\end{proof}
\end{st}

\subsection{Toevoegen van knooppunten en `de Boor' punten}
Wanneer we aan een gegeven splinecurve een punt willen toevoegen, zonder dat de curve verandert, moeten we \'e\'en extra knooppunt $u_l$ voorzien en de nieuwe controlepunten $d_i$ zoeken. 

\subsubsection*{Nieuwe knooppunten}
We voegen \'e\'en (vrij te kiezen) knooppunt $u_{l+1}^*$ toe aan de oude knooppunten $u_i$. 
\[
\begin{array}{c||c|c|c|c|c|c|c|c|c||}
\text{oud} & u_{l-n} & u_{l-n+1} & \cdots & u_l & & u_{l+1} & u_{l+2} & \cdots & u_{p+n}
\\
\text{nieuw} & u_{l-n}^* & u_{l-n+1}^* & \cdots & u_l^* & u_{l+1}^{*} & u_{l+2}^{*} & u_{l+3}^{*} & \cdots & u_{p+n+1}
\end{array}
\]

\subsubsection*{Nieuwe controlepunten}
De nieuwe `de Boor punten' vinden we dan als volgt.
\[
d_{i}^{*}
\left\{
\begin{array}{c}
d_{i} \text{ voor } i\le l\\
d_{i}^{[1]} \text{ voor } l-n< i\le l\\
d_{i-1} \text{ voor } i \ge l+1
\end{array}
\right.
\]
Schematisch ziet dat er als volgt uit. Merk op dat de onderste rij wel degelijk $1$ element meer bevat. Het extra element zit in de middenste "$\cdots$".
\[
\begin{array}{c||c|c|c|c|c|c|c|c|c|c|c|c|c||}
\text{oud} & d_{-n} & \cdots & d_{l-n-1} & d_{l-n} & d_{l-n+1} & d_{l-n+2} & \cdots & \cdots& d_{l-1} & d_{l} & d_{l+1} & \cdots & d_{p+1}
\\
&&&&\searrow&\downarrow\searrow&\downarrow\searrow & & \searrow & \downarrow\searrow & \downarrow&&&
\\
\text{nieuw} & d_{-n} & \cdots & d_{l-n-1} & d_{l-n}& d_{l-n+1}^{[1]} & d_{l-n+2}^{[1]} & \cdots & \cdots & d_{l-1}^{[1]} & d_{l}^{[1]} & d_{l+1} & \cdots& d_{p+1}
\end{array}
\]
De curve kunnen we dan in functie van de nieuwe `de Boor punten' en de nieuwe knooppunten schrijven.

\[
s(u) = \sum_{-n}^{p-1}d_iN_{i}^{n}(u) \text{ en } s^{*}(u) = \sum_{-n}^{p^{*}-1}d^{*}_iN_{i}^{*n}(u) 
\]
Merk op dat de $d_{\cdot}^{*}$, de nieuwe `de Boor punten' precies de punten zijn die we vinden in de eerste iteratie van het algoritme van de Boor.

\section{NURBS: rationale splinecurven}
\subsection{Rationale B\'ezier-curven}
Geef elk controlepunt een bepaald belang $w_i$ en defini\"eer een rationale B\'ezier-curve als volgt.
\begin{equation}
x(t) = \cfrac{\sum_{i=0}^{n}w_ib_iB_{i}^{n}(t)}{\sum_{i=0}^{n}w_iB_{i}^{n}(t)}
\end{equation}

\subsection{Rationale Splinecurven}
Geef elk controlepunt een bepaald belang $w_i$ en defini\"eer een rationale Splinecurve als volgt.
\[
x(u) = \cfrac{\sum_{i=-n}^{p-1}w_id_iN_{i}^{n}(u)}{\sum_{i=-n}^{p-1}w_iN_{i}^{n}(u)}
\]

\section{Samenvatting}
Voor een B-splinecurve van graad $n$:
\begin{itemize}
\item $p+1+2n$ knooppunten $u_{-n},\cdots,u_{-1},u_{0},\cdots,u_{p},u_{p+1},\cdots,u_{p+n}$.
\item $p+n$ `de Boor-punten' $d_{-n},\cdots,d_{0},\cdots,d_{p-1}$
\end{itemize}

\subsection{Verband tussen knooppunten, controlepunten en de splinecurve}
\begin{itemize}
\item $n+1$ knooppunten $u_{i},\cdots,u_{i+n+1}$ bepalen $N_{i}^{n}$.
\item $d_i$ be\"invloedt $s(u)$ voor $u\in [u_i,u_{i+n+1}]$ (een interval met $n+1$ knooppunten).
\item $s(u)$ hangt af van $n$ controlepunten $d_{j-n},\cdots,d_{j}$ als $u\in [u_{j},u_{j+1}]$.
\end{itemize}
\end{document}

\todo{invloed van samenvallende knooppunten en controlepunten}
