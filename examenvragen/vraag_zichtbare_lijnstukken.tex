\documentclass[examenvragen.tex]{subfiles}

\begin{document}

\section{Zichtbare lijnstukken}
\subsection{Opgave}
Gegeven een punt $P$ en $N$ lijnstukken $l_i$ die onderling niet snijden. Ontwerp een algoritme om alle zichtbare lijnstukken te vinden.

\subsection{Antwoord}
We kunnen dit realiseren in $O(N\log N)$ tijd met behulp van een sweepline algoritme.

Gebruik de eindpunten van de lijnstukken als event points.
Sorteer de event points volgens stijgende poolhoek.
Noem het eerste event point een toevoegpunt en het laatste een verwijderpunt.
Omdat het sorteren al $O(N\log N)$ tijd kost, moet het overlopen van de event points in maximum $O(\log N)$ tijd kunnen gebeuren per punt.
Dit kan inderdaad. We moeten hiervoor eerst een totale ordening defini\"eren van de lijnstukken in de status. We ordenen de lijnstukken volgens afstand tot $P$ van het toevoegpunt
Dit kan in $O(\log N)$ tijd gebeuren met een priority queue omdat die ordening niet verandert.
De status is een gebalanceerde binaire zoekboom gesorteerd op de afstand van het snijpunt met de sweepline tot $P$.
Hou bovendien een voorlopige resultaten-lijst $R$ bij.
Bij elk event point, voeg het toe aan $R$ als het zichtbaar is.
Dit doen we in constante tijd door te bekenen of het lijnstuk tussen $P$ en het toevoegpunt snijdt met het meest linkse lijnstuk in de status. 
Wanneer alle event points overlopen zijn zitten alle zichtbare event points in $R$. Wanneer zowel het begin als het eindpunt van een lijnstuk in $R$ zitten, noemen we het lijnstuk zichtbaar. 


\end{document}
