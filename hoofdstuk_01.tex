\documentclass[tmi_notities.tex]{subfiles}
\begin{document}
\chapter{Inleiding}
\section{Combinaties van punten}
Punten kunnen niet bij elkaar opgeteld worden, en niet gescaleerd worden omdat er geen oorsprong is en geen assenstelsel in een affiene ruimte.
Het verschil van twee punten is wel gedefinieerd, namelijk als een vector. Bovendien is de som van een punt en een vector ook gedefinieerd.
\subsection{Affiene combinatie}
Een affiene combinatie van punten is een lineaire combinatie van punten waarvan de gewichten sommeren tot $1$.\\\\
Zij $p_1,...p_n$ $n$ punten en $\sum_{i=1}^n\alpha_i=1$.\\\\
\textit{Te Bewijzen}\\
\label{affiene_combinatie}
Er bestaan $v_i$ zodat volgende bewering geldt.
\[
x = \sum_{i=1}^n\alpha_ip_i = p_0 + \sum_{i=2}^n\vec{v_i}
\]

\begin{proof}
We zullen dit bewijzen door de $v_i$ te construeren.
\[
\sum_{i=1}^n\alpha_ip_i
= \left(1-\sum_{i=2}^n\alpha_i\right)p_1 + \sum_{i=2}^n\alpha_ip_i
= p_1 - \sum_{i=2}^n\alpha_ip_1 + \sum_{i=2}^n\alpha_ip_i
\]
\[
=
p_1 + \sum_{i=2}^n\alpha_i(p_i-p_1)
\]
We weten dat $p_i-p_1$ een vector is. Sterker nog, het zijn de $v_i$.
\end{proof}
 
\subsection{Lineaire combinatie}
In \ref{affiene_combinatie} is nu eenvoudig te zien dat een lineaire combinatie van punten niet bepaald is.
Opdat de combinatie te herschijven zou vallen als de sum van een punt en een aantal vectoren moeten de gewichten van de lineaire combinatie sommeren tot $1$. 

\subsection{Convexe combinatie}
\textit{Te Bewijzen}\\
Een convexe combinatie van punten behoort tot de convex omhullende van deze punten.
\begin{proof}
Bewijs door inductie op het aantal punten ($n$).\\
Voor $n=1$ is de convex omhullende enkel dat punt en elke convexe combinatie van een punt is dat punt.\\
Veronderstel dat de bewering geldt voor een bepaalde $n=k$. We bewijzen nu daaruit volgt dat de bewering geldt voor $n=k+1$.\\
Zij $x$ de convexe combinatie van de punten $p_1,...,p_k,p_{k+1}$.
\[
x = \sum_{i=1}^{k+1}\alpha_ip_i
\]
Als $\alpha_{k+1} = 1$ geldt dan is de bewering triviaal. In wat volgt gaan we ervan uit dat $\alpha_{k+1} \neq 1$ geldt.
We weten nu dat volgende bewering geldt, omdat het om een convexe combinatie gaat.
\[
\sum_{i=1}^{k}\alpha_i = S \rightarrow 0 \le S \le 1
\]
Beschouw nu een punt $y$ dat als volgt gedefinieerd is met $\beta_i = \frac{\alpha_i}{S}$.
\[
y = \sum_{i=1}^k\beta_{i}p_{i}
\]
We sommeren nu de $\beta_i$ om te zien dat $y$ ook een convexe combinatie is.
\[
\sum_{i=1}^k\beta_{i} = \sum_{i=1}^k \frac{\alpha_i}{S} = \frac{1}{S}\sum_{i=1}^k \alpha_i = \frac{\sum_{i=1}^{k}\alpha_i}{\sum_{i=1}^{k}\alpha_i} = 1
\]
Volgens de inductiehypothese ligt $y$ in de convex omhullende van $p_1,...,p_k$. $y$ ligt dus zeker in de convex omhullende van $p_1,...,p_k,p_{k+1}$. Nu geldt het volgende over $x$ met $1-S = \alpha_{k+1}$.
\[
x = Sy + (1-S)p_{k+1}
\]
Dit is opnieuw een convexe combinatie, van twee punten binnen de convex omhullende. $x$ behoort dus tot de convex omhullende van $p_1,...,p_k,p_{k+1}$. 
\end{proof}
\end{document}

