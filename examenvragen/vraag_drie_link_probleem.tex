\documentclass[examenvragen.tex]{subfiles}

\begin{document}

% Index, x, y, r
\def\circle[#1,#2,#3,#4] {
  % The middlepoint
  \coordinate (C#1) at (#2,#3);

  % The circle
  \draw[thick,color=black] (C#1) circle (#4);
  
}

\section{Drie linken}
\subsection{Generisch}
\subsubsection{Opgave}
Gegeven een robotarm met drie segmenten. De segmenten hebben een respectievelijke lengte van $l_1$, $l_2$ en $l_3$. Bepaal de optimale manier om een punt $p$ op een afstand $l$ van de `schouder' van de robot te bereiken.

\subsubsection{Antwoord}
Elk drie linken $(l_1,l_2,l_3)$ kan herleid worden tot \'e\'en van de volgende twee linken problemen.
\begin{itemize}
\item $(l_1+l_2,l_3)$: Geen knik in het gewricht $j_1$ tussen de eerste twee segmenten $l_1$ en $l_2$.
\item $(l_1,l_2+l_3)$: Geen knik in het gewricht $j_2$ tussen de laatste twee segmenten $l_2$ en $l_3$.
\item $(l_2,l_3)$: Twee knikken, maar we kunnen de hoek in het eerste gewricht $j_0$ vrij kiezen. Gebruik het eindpunt van het eerste segment als nieuwe oorsprong.
\end{itemize}
Een twee linken $(s_1,s_2)$ probleem met punt $p$ valt in constante tijd op te lossen als volgt.
\begin{enumerate}
\item Teken een cirkel $c_1$ met straal $s_1$ in de oorsprong.
\item Teken een cirkel $c_2$ met straal $s_2$ in $p$.
\item Bereken de snijpunten $i$ van de twee cirkels $c_1$ en $c_2$. Elk snijpunt $i$ is een mogelijke locatie voor de knik. 
\end{enumerate}

\begin{figure}[H]
  \centering
  \resizebox {0.3\textwidth} {!} {
    \begin{tikzpicture}[scale=1.5]
      % grid
      \draw[very thin,color=lightgray] (0,0) grid (3,3);
      \draw [<->,thick] (0,3) node (yaxis) [left] {Y} |- (3,0) node (xaxis) [right] {X};
      
      \circle[1,1,1.5,1]
      \circle[2,2,2,1]    

	  % The point
  	\draw[fill,color=black] (C1) circle (1.5pt) 
  	node[left, yshift=-10pt, color=black] {$O$};
  	\draw[fill,color=black] (C2) circle (1.5pt) 
  	node[left, yshift=-10pt, color=black] {$p$};
  
  	\coordinate (j1) at (1.129190075645217, 2.491619848709566);
  	\draw[fill,color=black] (j1) circle (1.5pt) 
  	node[left, yshift=10pt, color=black] {$s_1$};
  	
  	\draw[dashed] (C1) -- (j1)
  	node[left, yshift=-15pt, xshift=-5pt, color=black] {$s_1$};
  	
  	\draw[dashed] (j1) -- (C2)
  	node[left, yshift=0pt, xshift=-15pt, color=black] {$s_1$};
      
    \end{tikzpicture}
  }
\end{figure}

\subsection{Voorbeeld}
\subsubsection{Opgave}
Gegeven een robotarm met drie segmenten. De segmenten hebben een respectievelijke lengte van $30$, $20$ en $40$. Bepaal de optimale manier om een punt $p$ op een afstand $45$ van de `schouder' van de robot te bereiken.


\subsubsection{Antwoord}
We herleiden dit probleem tot een twee linken probleem van de eerste soort $(l_1+l_2,l_3)$.
Het twee linken probleem $(50,40)$ kunnen we eenvoudig per constructie oplossen. Teken een cirkel met straal $50$ rond de oorsprong en een cirkel met straal $40$ rond $p$. Duid de snijpunten aan.



\end{document}
