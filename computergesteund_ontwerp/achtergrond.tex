\documentclass[computergesteund_ontwerp_van_curven_en_oppervlakken.tex]{subfiles}
\begin{document}

\chapter{Achtergond}
\section{De affiene ruimte}
Een affiene ruimte is een tupel $(A,V)$ van een verzameling $A$ en een vectorruimte $V$. Op $A$ is een optelling met $V$ gedefinie\"erd.
\[
l: V \times A \rightarrow A: \vec{v}+a
\]
Een affiene ruimte heeft de volgende eigenschappen.
\begin{enumerate}
\item Linker Identiteit
\[
\forall a \in A: \vec{0} + a = a
\]
\item Associativiteit
\[
\forall v, w \in V, \forall a \in A,\; v + (w + a) = (v + w) + a
\]
\item Uniciteit
\[
\forall a \in A,\; V \to A\colon v \mapsto v + a \text{ is een bijectie}\]
\end{enumerate}
Vanwege de uniciteitseigenschap is de aftrekking van twee elementen $a_1$ en $a_2$ gedefini\"eerd. In de rest van deze notities zal de verzameling $A$ een verzameling punten zijn en $V$ een verzameling verschilvectoren.


\section{Combinaties van punten}
Punten kunnen niet bij elkaar opgeteld worden, en niet gescaleerd worden omdat er geen oorsprong is en geen assenstelsel in een affiene ruimte.
Het verschil van twee punten is wel gedefinieerd, namelijk als een vector. Bovendien is de som van een punt en een vector ook gedefinieerd.
\subsection{Affiene combinatie}
Een affiene combinatie van punten is een lineaire combinatie van punten waarvan de gewichten sommeren tot $1$.
\subsubsection{Een cruciaal inzicht}
Zij $p_1,...p_n$ $n$ punten en $\alpha_i$, scalars zodat $\sum_{i=1}^n\alpha_i=1$ geldt.\\\\
\textit{Te Bewijzen}\\
\label{affiene_combinatie}
De affiene combinatie van de $p_i$ kan gezien worden als de optelling van $n-1$ vectoren bij het eerste punt.
Er bestaan $v_i$ zodat volgende bewering geldt.
\[
x = \sum_{i=1}^n\alpha_ip_i = p_0 + \sum_{i=2}^n\vec{v_i}
\]

\begin{proof}
We zullen dit bewijzen door de $v_i$ te construeren.
\[
\sum_{i=1}^n\alpha_ip_i
= \left(1-\sum_{i=2}^n\alpha_i\right)p_1 + \sum_{i=2}^n\alpha_ip_i
= p_1 - \sum_{i=2}^n\alpha_ip_1 + \sum_{i=2}^n\alpha_ip_i
\]
\[
=
p_1 + \sum_{i=2}^n\alpha_i(p_i-p_1)
\]
We weten dat $p_i-p_1$ een vector is. Sterker nog, het zijn de $v_i$.
\end{proof}

\subsection{Affiene transformatie}
\label{affiene_transformatie}
Een affiene transformatie is een transformatie die elke affiene combinatie van punten afbeeldt op diezelfde affiene combinatie van punten.\\
In symbolen: Zij $T$ een affiene transformatie en $x$ een punt gedefinieerd als een affiene combinatie.
\[
x = \sum_{i=1}^{n}\alpha_ip_i
\]
Voor $T$ geldt dan de volgende bewering.
\[
T(x) = \sum_{i=1}^{n}\alpha_iT(p_i)
\]
 
\subsection{Lineaire combinatie}
In \ref{affiene_combinatie} is nu eenvoudig te zien dat een lineaire combinatie van punten niet bepaald is.
Opdat de combinatie te herschijven zou vallen als de sum van een punt en een aantal vectoren moeten de gewichten van de lineaire combinatie sommeren tot $1$. 

\subsection{Convexe combinatie}
\label{convexe_combinatie}
\textit{Te Bewijzen}\\
Een convexe combinatie van punten behoort tot de convex omhullende van deze punten.
\begin{proof}
Bewijs door inductie op het aantal punten ($n$).\\
Voor $n=1$ is de convex omhullende enkel dat punt en elke convexe combinatie van een punt is dat punt.\\
Veronderstel dat de bewering geldt voor een bepaalde $n=k$. We bewijzen nu dat daaruit volgt dat de bewering geldt voor $n=k+1$.\\
Zij $x$ de convexe combinatie van de punten $p_1,...,p_k,p_{k+1}$.
\[
x = \sum_{i=1}^{k+1}\alpha_ip_i
\]
Als $\alpha_{k+1} = 1$ geldt dan is de bewering triviaal. In wat volgt gaan we ervan uit dat $\alpha_{k+1} \neq 1$ geldt.
We weten nu dat volgende bewering geldt, omdat het om een convexe combinatie gaat.
\[
\sum_{i=1}^{k}\alpha_i = S \rightarrow 0 \le S \le 1
\]
Beschouw nu een punt $y$ dat als volgt gedefinieerd is met $\beta_i = \frac{\alpha_i}{S}$.
\[
y = \sum_{i=1}^k\beta_{i}p_{i}
\]
We sommeren nu de $\beta_i$ om te zien dat $y$ ook een convexe combinatie is.
\[
\sum_{i=1}^k\beta_{i} = \sum_{i=1}^k \frac{\alpha_i}{S} = \frac{1}{S}\sum_{i=1}^k \alpha_i = \frac{\sum_{i=1}^{k}\alpha_i}{\sum_{i=1}^{k}\alpha_i} = 1
\]
Volgens de inductiehypothese ligt $y$ in de convex omhullende van $p_1,...,p_k$. $y$ ligt dus zeker in de convex omhullende van $p_1,...,p_k,p_{k+1}$. Nu geldt het volgende over $x$ met $1-S = \alpha_{k+1}$.
\[
x = Sy + (1-S)p_{k+1}
\]
Dit is opnieuw een convexe combinatie, van twee punten binnen de convex omhullende. $x$ behoort dus tot de convex omhullende van $p_1,...,p_k,p_{k+1}$. 
\end{proof}

\subsection{Interpolerende veeltermen}
\subsubsection{Lagrange veelterm}
Bij $n+1$ punten $p_i$ horen de volgende $n$ Lagrange veeltermen $L_i$.
\[
L_{i}^{n}(x) = \prod_{j=0, j \neq i}^{n} \frac{x-x_{j}}{x_{i}-x_{j}}
\]
Lagrange veeltermen hebben een aantal handige eigenschappen.
\begin{enumerate}
\item Voor elke $i$ geldt volgende formule geldt:
\[
L_{i}^{n}(x_i) = 1
\]
\begin{proof}

\[
L_{i}^{n}(x_i)
= \prod_{j=0, j \neq i}^{n} \frac{x_i-x_{j}}{x_{i}-x_{j}}
= \prod_{j=0, j \neq i}^{n} 1
= 1
\]
\end{proof}
\item Voor elk ander punt $k$ geldt bovendien dit: ($i\neq k$)
\[
L_i^{n}(x_k) = 0
\]
\begin{proof}
\[
L_{i}^{n}(x_k)
= \prod_{j=0, j \neq i}^{n} \frac{x_k-x_{j}}{x_{i}-x_{j}}
= \frac{x_k-x_{1}}{x_{i}-x_{1}}\cdot \ldots \cdot 0 \cdot \ldots \cdot \frac{x_k-x_{n}}{x_{i}-x_{n}}
= 0
\]
\end{proof}
\item De som van alle $n+1$ lagrange veeltermen is steeds $1$.
\[
\forall t: \sum_{i=0}^{n}L_i^{n}(t) = \sum_{i=0}^{n}\prod_{j=0, j \neq i}^{n} \frac{t-x_{j}}{x_{i}-x_{j}} = 1
\]
\begin{proof}
Stel de interpolerende veelterm van graad $n$ van de constante functie $y=1$ op.
\[
y_n = \sum_{i=0}^{n} f(x_i) \prod_{j=0,j\neq i}^{n}\frac{x-x_j}{x_i-x_j}
\]
We weten nu dat elke $f(x_i)$ gelijk is aan $1$ en dat $y_n$ $y$ exact benadert.
Bijgevolg geldt het volgende.
\[
\sum_{i=0}^{n}\prod_{j=0,j\neq i}^{n}\frac{x-x_j}{x_i-x_j} = 1
\]
\end{proof}

\end{enumerate}

\end{document}
