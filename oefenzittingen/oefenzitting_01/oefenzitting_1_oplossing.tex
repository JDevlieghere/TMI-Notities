\documentclass[10pt,a4paper]{article}
\usepackage[latin1]{inputenc}
\usepackage[dutch]{babel}
\usepackage{amsmath}
\usepackage{amsfonts}
\usepackage{amssymb}
\usepackage{amsthm}
\usepackage{pdfpages}

\title{Oefenzitting 1}
\author{Tom Sydney Kerckhove}
\date{3 maart 2014}

\begin{document}
\maketitle

\section{Vraag 1}
\begin{proof}
\[
\frac{d}{dt}B^{n}_{i}(t)
\]
\[
= \frac{d}{dt}\binom{n}{i}t^i(1-t)^{n-i}
\]
\[
= \binom{n}{i}\frac{d}{dt}t^i(1-t)^{n-i}
\]
\[
= \binom{n}{i}
\left(
(1-t)^{n-i}it^{i-1} - t^i(n-i)(1-t)^{n-i-1}
\right)
\]
\[
=
\left(
\binom{n}{i}(1-t)^{n-i}it^{i-1} - \binom{n}{i}t^i(n-i)(1-t)^{n-i-1}
\right)
\]
\[
=
\left(
\frac{n!}{i!(n-i)!}(1-t)^{n-i}it^{i-1} - \frac{n!}{i!(n-i)!}t^i(n-i)(1-t)^{n-i-1}
\right)
\]
\[
=
\left(
\frac{n!}{(i-1)!(n-i)!}(1-t)^{n-i}t^{i-1} - \frac{n!}{i!(n-i-1)!}t^i(1-t)^{n-i-1}
\right)
\]
\[
=
n
\left(
\frac{(n-1)!}{(i-1)!(n-i)!}(1-t)^{n-i}t^{i-1} - \frac{(n-1)!}{i!(n-i-1)!}t^i(1-t)^{n-i-1}
\right)
\]
\[
=
n
\left(
B^{n-1}_{i-1}(t) - B^{n-1}_{i}(t)
\right)
\]
\end{proof}
\[
n
\left(
B^{n-1}_{i-1}(t) - B^{n-1}_{i}(t)
\right)
=0
\]
\[
\Leftrightarrow
\] 
\[
n=0 \wedge B^{n-1}_{i-1}(t) - B^{n-1}_{i}(t) = 0
\]
\[
B^{n-1}_{i-1}(t) = B^{n-1}_{i}(t)
\]
\[
\frac{(n-1)!}{(i-1)!(n-i)!}(1-t)^{n-i}t^{i-1} = \frac{(n-1)!}{i!(n-i-1)!}t^i(1-t)^{n-i-1}
\]
\[
\frac{(1-t)}{(n-i)} = \frac{t}{i}
\]
\[
i(1-t)=t(n-i) 
\]
\[
i-it-tn+it= i+it =0
\]
\[
i=0 \wedge t=-1
\]

\section{Vraag 2}
Zij:
\[
\vec{x}(t) = \vec{x}(t(u)) = \sum_{i=0}^{n}\vec{b_{i}}B_{i}^{n-1}(t)
\]
\subsection*{(a)}
Te bewijzen:
\[
\frac{d}{dt}\vec{x}(t) = n\sum_{i=0}^{n-1}\Delta\vec{b_i}B_{i}^{n-1}(t)
\]
\begin{proof}
\[
\frac{d}{dt}\vec{x}(t)
= \frac{d}{dt}\sum_{i=0}^{n}\vec{b_{i}}B_{i}^{n}(t)
= \sum_{i=0}^{n}\vec{b_{i}} \frac{d}{dt}B_{i}^{n}(t)
\]
\[
= \sum_{i=0}^{n}\vec{b_{i}}n
\left(
B^{n-1}_{i-1}(t) - B^{n-1}_{i}(t)
\right)
\]
\[
= n
\sum_{i=1}^{n}
\vec{b_{i}}B^{n-1}_{i-1}(t)
-n 
\sum_{i=0}^{n-1}
\vec{b_{i}}B^{n-1}_{i}(t)
\]
Verander de index van de eerste sommatie (hoog ze met 1 op)
\[
= n
\sum_{i=0}^{n-1}
\vec{b_{i+1}}B^{n-1}_{i}(t)
-n 
\sum_{i=0}^{n-1}
\vec{b_{i}}B^{n-1}_{i}(t)
\]
\[
= n
\sum_{i=0}^{n-1}
(\vec{b_{i+1}}-\vec{b_{i}})B^{n-1}_{i}(t)
\]
\[
= n
\sum_{i=0}^{n-1}\Delta\vec{b_{i}}B^{n-1}_{i}(t)
\]
\end{proof}

\subsection*{(b)}
\[
n(n-1)
\sum_{i=0}^{n-2}\Delta\vec{b_{i}}B^{n-2}_{i}(t)
\]

\subsection*{(c)}
De eerste uitdrukking behandelt het curvesegment, de tweede uitdrukking de volledige curve.


\subsection*{(d)}
De eerste 3. De tweede afgeleide zegt iets over kromming van de curve.

\section{Vraag 3}
Zij:
\[
\vec{x}(t) = \sum_{i=0}^{n}\vec{b_{i}}B^{n}_{i}(t) = \sum_{i=0}^{n}\vec{a_{i}}t^i
\]
Te bewijzen:
\[
\vec{a_i} = \binom{n}{i}\Delta^i\vec{b}_0
\]
\subsection*{(a)}
Zij $n=3$:
\[
\sum_{i=0}^{n}\vec{b_{i}}B^{n}_{i}(t)
= \vec{b_{0}}B^{3}_{0}(t)
+ \vec{b_{1}}B^{3}_{1}(t)
+ \vec{b_{2}}B^{3}_{2}(t)
+ \vec{b_{3}}B^{3}_{3}(t)
\]
\[
= \vec{b_{0}}(1-t)^{3}
+ \vec{b_{1}}3(1-t)^{2}t^{1}
+ \vec{b_{2}}3(1-t)^{1}t^{2}
+ \vec{b_{3}}t^{3}
\]
\[
= \vec{b_{0}}(1-3t+3t^2-t^3)
+ \vec{b_{1}}3(1+2t+t^2)t
+ \vec{b_{2}}3(1-t)t^{2}
+ \vec{b_{3}}t^{3}
\]
\[
= t^3(- \vec{b_0}+3\vec{b_1}-3\vec{b_2}+\vec{b_3})
+ t^2( 3\vec{b_0}-6\vec{b_1}+3\vec{b_2})
+ t^1(-3\vec{b_0}+3\vec{b_1})
+ t^0   \vec{b_0}
\]
\[
=  t^3(-\vec{b_0}+3\vec{b_1}-3\vec{b_2}+\vec{b_3})
+ 3t^2( \vec{b_0}-2\vec{b_1}+ \vec{b_2})
+ 3t^1(-\vec{b_0}+ \vec{b_1})
+  t^0  \vec{b_0}
\]
\[
= \binom{3}{i}\Delta^i\vec{b}_0
\]

\subsection*{(b)}
\begin{proof}
Stel dat de bewering geldt voor een bepaalde $n=k$, dan bewijzen we nu dat de bewering dan ook geldt voor $n=k+1$.
\[
\sum_{i=0}^{k+1}\vec{b_{i}}B^{k+1}_{i}(t)
\]
WUT
\end{proof}

\section{Vraag 4}
Niet, er is minstens een vierdegraads B\'ezier-curve voor nodig.

\section{Vraag 5}
Maak de controlepunte lineair.
\begin{proof}
Stel dat er een derdegraads B\'ezier curve zou bestaan die een lijnstuk is, maar waarvan de controlepunten niet colineaire zijn. De curve moet raken aan de controle-veelhoek in $t=0$ en $t=1$. Deze controleveelhoek is geen lijnstuk, dus de curve moet ergens een bocht maken, maar dan stelt de curve geen lijnstuk meer voor. Contradictie.
\end{proof}

\section{Vraag 6}
\subsection*{(a)}
$O(n^3)$

\subsection*{(b)}
$O(n^2)$.

\subsection*{(c)}
$O(n^2)$ voorbereiding, maar $O(n)$ voor elk punt.

\section{Vraag 7}
NOPE

\includepdf[pages=-,landscape]{opgave.pdf}

\end{document}