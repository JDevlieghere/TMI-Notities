\documentclass[computergesteund_ontwerp_van_curven_en_oppervlakken.tex]{subfiles}
\begin{document}

\chapter{Splinecurven}
\section{Definities}
\subsection{Splinecurve}
\[
\vec{s}(u) = \sum_{i=-n}^{p-1}\vec{d_i}N_{i}^{n}(u)
\]
Hierbij zijn de $\vec{d_i}$ de `de Boor' punten.

\subsection{B-Spline}
\[
N_{i}^{n}(u)
= \frac{u-u_i}{u_{i+n}-u_i}					N_{i}^{n-1}(u)
+ \frac{u_{i+n+1}-u}{u_{i+n+1}-u_{i+1}}		N_{i+1}^{n-1}(u)
\]
met
\[
u \in [u_i,u_{i+1}) \Rightarrow N_{i}^{0}(u) = 1
\]
\[
u \not\in [u_i,u_{i+1}) \Rightarrow N_{i}^{0}(u) = 0
\]

\section{Eigenschappen van B-Splines}
\begin{enumerate}
\item Lokaliteit
\[
u \not\in [u_i,u_{i+n+1}] \Rightarrow N_{i}^{n}(u) = 0
\]

\begin{proof} Bewijs door inductie.\\
De bewering geldt voor $n=0$.
\[
u \in [u_i,u_{i+1}) \Rightarrow N_{i}^{0}(u) = 1
\]
\[
u \not\in [u_i,u_{i+1}) \Rightarrow N_{i}^{0}(u) = 0
\]
We bewijzen nu dat als de bewering geldt voor een bepaalde $n=k$, de bewering dan ook geldt voor $n=k+1$.
Te bewijzen is nu het volgende.
\[
u \not\in [u_i,u_{i+k+2}] \Rightarrow N_{i}^{k+1}(u) = 0
\]
We weten het volgende over $N_{i}^{k+1}(u)$.
\[
N_{i}^{k+1}(u)
= \frac{u-u_i}{u_{i+k+1}-u_i}					N_{i}^{k}(u)
+ \frac{u_{i+k+2}-u}{u_{i+k+2}-u_{i+1}}			N_{i+1}^{k}(u)
\]
Deze uitdrukking is nul wanneer $N_{i}^{k}(u)$ en $N_{i+1}^{k}(u)$ beide nul zijn.
Vanwege de inductiehypothese is dit zo omdat de volgende beweringen gelden.
\[
[u_i,u_{i+k+1}] \subset [u_i,u_{i+k+2}]
\text{ en }
[u_{i+1},u_{i+k+2}] \subset [u_i,u_{i+k+2}]
\]
\end{proof}

\item Positiviteit
\[
N_{i}^{n} \ge 0
\]
\begin{proof} Bewijs door inductie.\\
De bewering geldt voor $n=0$ omdat $N_{i}^{0}$ enkel nul of \'e\'en kan zijn.
We bewijzen nu dat als de bewering geldt voor een bepaalde $n=k$, de bewering dan ook geldt voor $n=k+1$.
We bekijken de uitdrukking voor $N_{i}^{k+1}(u)$.
\[
N_{i}^{k+1}(u)
= \frac{u-u_i}{u_{i+k+1}-u_i}					N_{i}^{k}(u)
+ \frac{u_{i+k+2}-u}{u_{i+k+2}-u_{i+1}}			N_{i+1}^{k}(u)
\]
Deze uitdrukking zou enkel negatief kunnen zijn als $u < u_i$ of $u_{i+n+1} < u$ geldt. (uit de inductiehypothese volgt dat $N_{i}^{k}(u)$ en $N_{i+1}^{k}(u)$ positief zijn.)
$N_{i}^{k+1}(u)$ is echter steeds nul wanneer $u \not\in [u_i,u_{i+k+2}]$ en binnen $[u_i,u_{i+k+2}]$ is $u$ nooit kleiner dan $u_i$ of groter dan $u_{i+k+2}$.
\end{proof}

\item Eenheidspartitie (Sommatie tot $1$)
\[
\forall u \in [u_0,u_p)\ :\
\sum_{i=-\infty}^{\infty}N_{i}^{n}(u)
= \sum_{i=-n}^{p-1}N_{i}^{n}(u)
= 1
\]

\begin{proof}Bewijs door inductie.\\
De bewering geldt voor $n=0$ want voor elk interval $u_j,u_{j+1}$ is er precies \'e\'en $N_{i}^{0}$ die gelijk is aan \'e\'en terwijl de andere nul zijn.
\[
u \in [u_0,u_p) \Rightarrow
\sum_{i=0}^{p-1}N_{i}^{0}(u)
= 1
\]
We bewijzen nu dat als de bewering geldt voor een bepaalde $n=k$, de bewering dan ook geldt voor $n=k+1$. We stellen dus dat volgende bewering geldt.
\[
\forall u \in [u_0,u_p) \Rightarrow
\sum_{i=-k}^{p-1}N_{i}^{k}(u)
= 1
\]
Nu bewijzen we dat de volgende bewering daaruit volgt.
\[
\forall u \in [u_0,u_p) \Rightarrow
\sum_{i=-k-1}^{p-1}N_{i}^{k+1}(u)
= 1
\]
\[
\sum_{i=-k-1}^{p-1}N_{i}^{k+1}(u)
= \sum_{i=-k-1}^{p-1}
\left(
\frac{u-u_i}{u_{i+k+1}-u_i}					N_{i}^{k}(u)
+ \frac{u_{i+k+2}-u}{u_{i+k+2}-u_{i+1}}			N_{i+1}^{k}(u)
\right)
\]
\[
=
\sum_{i=-k-1}^{p-1}
\frac{u-u_i}{u_{i+k+1}-u_i}					N_{i}^{k}(u)
+ 
\sum_{i=-k-1}^{p-1}
\frac{u_{i+k+2}-u}{u_{i+k+2}-u_{i+1}}			N_{i+1}^{k}(u)
\]
In deze uitdrukking zijn de eerste en de laatste term beide nul vanwege de lokaliteitseigenschap.
\[
=
\sum_{i=-k}^{p-1}
\frac{u-u_i}{u_{i+k+1}-u_i}					N_{i}^{k}(u)
+ 
\sum_{i=-k-1}^{p-2}
\frac{u_{i+k+2}-u}{u_{i+k+2}-u_{i+1}}			N_{i+1}^{k}(u)
\]
De tweede som kan herschreven worden door de index aan te passen als volgt.
\[
\sum_{i=-k-1}^{p-2}
\frac{u_{i+k+2}-u}{u_{i+k+2}-u_{i+1}}			N_{i+1}^{k}(u)
=
\sum_{i=-k}^{p-1}
\frac{u_{i+k+1}-u}{u_{i+k+1}-u_{i}}			N_{i}^{k}(u)
\]
We vervangen dit nu zodat we verder gaan met de volgende uitdrukking.
\[
=
\sum_{i=-k}^{p-1}
\frac{u-u_i}{u_{i+k+1}-u_i}					N_{i}^{k}(u)
+ 
\sum_{i=-k}^{p-1}
\frac{u_{i+k+1}-u}{u_{i+k+1}-u_{i}}			N_{i}^{k}(u)
\]
\[
=
\sum_{i=-k}^{p-1}
\left(
\frac{u-u_i}{u_{i+k+1}-u_i}					N_{i}^{k}(u)
+ 
\frac{u_{i+k+1}-u}{u_{i+k+1}-u_{i}}			N_{i}^{k}(u)
\right)
\]
\[
=
\sum_{i=-k}^{p-1}
\frac{u-u_i + u_{i+k+1}-u}{u_{i+k+1}-u_{i}} N_{i}^{k}(u)
=
\sum_{i=-k}^{p-1}
N_{i}^{k}(u)
\]
Uit de inductiehypothese volgt precies dat deze uitdrukking gelijk is aan \'e\'en. Hiermee is de stelling bewezen.
\end{proof}

\item Continu\"iteit.\\
$N_{i}^{n}(u)$ is een $C_{n-1}$ continue functie indien er geen meervoudige knooppunten voorkomen. \footnote{De continu\"iteitseigenschap kan niet bewezen worden via de recursiebetrekking. Wanneer we B-splines op een andere manier (equivalent) defini\"eren valt dit wel te bewijzen.}
\end{enumerate}

\section{Eigenschappen van B-spline curven}
\begin{enumerate}
\item Spline curves zijn invariant onder affiene transformaties.
\begin{proof}
Dit volgt uit de eenheidspartitie eigenschap van B-splines. Een B-spline curve is een affiene combinatie van de `de Boor' punten $\vec{d_i}$. Bijgevolg is een spline curve invariant onder een affiene transformatie (Zie \ref{affiene_transformatie}.).
\end{proof}

\item Bij wijziging van \'e\'en `de Boor' punt $\vec{d_i}$ zal niet heel de curve veranderen. Enkel het deel dat overeenstemt met $u \in [u_i,u_{i+n+1})$.
\begin{proof}
Dit volgt uit de lokaliteitseigenschap van B-splines. $B$ splines buiten het interval $[u_i,u_{i+n+1})$ zijn namelijk steeds nul. De invloed van $\vec{d_i}$ in buiten $[u_i,u_{i+n+1})$ is dus onbestaande.
\end{proof}

\item
Spline curves liggen binnen de convex omhullende van de `de Boor' punten. Sterker nog, de curve ligt steeds binnen de convex omhullende van de $n+1$ omliggende `de Boor' punten ligt.

\begin{proof}
In het deel van de curve dat overeenstemt met $u \in [u_j,u_{j+1})$ zijn enkel de B-splines $N_{j-n}^{n},..., N_{j}^{n}$ niet nul. Vanwege de positiviteitseigenschap zijn deze B-splines allemaal positief en vanwege de eenheidspartitieeigenschap sommeren ze tot \'e\'en. Een punt op de curve is dus een convexe combinatie van $n+1$ omliggende `de Boor' punten. Bovendien ligt elke convexe combinatie van punten binnen de convex omhullende van die punten (Zie \ref{convexe_combinatie}.).
\end{proof}
\end{enumerate}

\section{Evaluatie van een punt op een B-Spline curve (de Boor)}
Gegeven een curve $s(u)$ gedefinie\"erd over een interval $u_0,u_p$. 
\[
s(u) = \sum_{i=-n}^{p-1}\vec{d_i}N_{i}^{n}(u)
\]
Gevraagd is het punt op $s$ dat overeen komt met $u\in [u_l,u_{l+1})$.
Door de lokaliteitseigenschap weten we dat het punt afhangt van de `de Boor' punten $\vec{d^{l-n}},...,\vec{d_{l}}$, de andere `de Boor' punten spelen geen rol.\\
Definieer $\vec{d_{i}^{[k]}}$ als volgt.
\[
\vec{d_{i}^{[k]}}
= (1-\alpha_{k,i})\vec{d_{i-1}^{[k-1]}}
+ \alpha_{k,i}\vec{d_{i}^{[k-1]}}
\text{ met }
\alpha_{k,i} = \frac{u-u_i}{u_{i+n+1-k}-u_{i}}
\]
$s(u)$ is nu precies $\vec{d_{l}^{[k]}}$.
\begin{proof}
\[
s(u)
= \sum_{i=-n}^{p-1}\vec{d_i}N_{i}^{n}(u)
\underset{u \in [u_l,u_{l+1})}{=}
\sum_{i=l-n}^{l}\vec{d_i}N_{i}^{n}(u)
\]
\[
=
\sum_{i=l-n}^{l}\vec{d_i}
\left(
\frac{u-u_i}{u_{i+n}-u_i}N_i^{n-1}(u) + 
\frac{u_{i+n+1}-u}{u_{i+n+1}-u_{i+1}}N_{i+1}^{n-1}(u)
\right)
\]
\[
=
\sum_{i=l-n}^{l}\vec{d_i}
\frac{u-u_i}{u_{i+n}-u_i}N_i^{n-1}(u) + 
\sum_{i=l-n}^{l}\vec{d_i}
\frac{u_{i+n+1}-u}{u_{i+n+1}-u_{i+1}}N_{i+1}^{n-1}(u)
\]
Bovendien geldt $N_{l-n}^{n-1}(u) = 0$ en $N_{l+1}^{n-1}(u) = 0$
\[
=
\sum_{i=l-n+1}^{l}\vec{d_i}
\frac{u-u_i}{u_{i+n}-u_i}N_i^{n-1}(u) + 
\sum_{i=l-n}^{l-1}\vec{d_i}
\frac{u_{i+n+1}-u}{u_{i+n+1}-u_{i+1}}N_{i+1}^{n-1}(u)
\]
De tweede som in bovenstaande uitdrukking valt te herschrijven als volgt.
\[
\sum_{i=l-n}^{l-1}\vec{d_i}
\frac{u_{i+n+1}-u}{u_{i+n+1}-u_{i+1}}N_{i+1}^{n-1}(u)
=
\sum_{i=l-n+1}^{l}\vec{d_{i-1}}
\frac{u_{i+n}-u}{u_{i+n}-u_{i}}N_{i}^{n-1}(u)
\]
We kunnen dit dus vervangen.
\[
s(u)
=
\sum_{i=l-n+1}^{l}
N_i^{n-1}(u)
\left(
\vec{d_{i-1}}
\frac{u-u_i}{u_{i+n}-u_i} +
\vec{d_i}
\frac{u_{i+n}-u}{u_{i+n}-u_{i}}
\right)
\]
\[
s(u)
= \sum_{i=l-n+1}^{l}
\vec{d_{i}^{[1]}}
N_i^{n-1}(u)
\]
Dit proces kunnen we nog $n-1$ maal herhalen, tot we volgende uitdrukking bekomen.
\[
s(u) = \sum_{i=l}^{l}\vec{d}_{i}^{[n]}N_{i}^{0}(u) = \vec{d}_{i}^{[n]}
\]
\end{proof}

\subsection{Toevoegen van knooppunten en `de Boor' punten}
Wanneer we \'e'en knooppunt en \'e\'en `de Boor' punt toevoegen (en we willen dat de curve hetzelfde blijft,) voeren we volgend proces uit.\\
We willen een splinecurve met knooppunten $u_0,...,u_p$ en `de Boor' punten $\vec{d_{-n}},...,\vec{d_{p-1}}$ te schrijven als een splinecurve met knooppunten $u^{*}_0,...,u^{*}_{p^{*}}$ en `de Boor' punten $\vec{d}_{-n}^{*},...,\vec{d}_{p^{*}-1}^{*}$.
\[
s(u) = \sum_{-n}^{p-1}\vec{d}_iN_{i}^{n}(u) \text{ en } s^{*}(u) = \sum_{-n}^{p^{*}-1}\vec{d}^{*}_iN_{i}^{*n}(u) 
\]
\todo{Volledig uitwerken, dit is redelijk vaag.}
De nieuwe knooppunten zien er dan als volgt uit.
\[
\{u_{-n},...,u_{p}\}
= \{u_{-n},...,u_{l},u_{l+1},...,u_{p^{*}}\} 
\]
De nieuwe `de Boor' punten zien er dan als volgt uit.
\[
\vec{d}_{l-n}^{*} = \vec{d}_{l-n} \text{, } \vec{d}_{l+1}^{*} = \vec{d}_{l}\text{ en } \vec{d}_{i}^{*} = \vec{d}_{i}^{[1]} \text{ voor } i\ge l-n+1
\]
\todo{Betere uitleg hier!!}

\section{NURBS: rationale splinecurven}
\todo{schrijf hier iets zinnings over}

\end{document}
