\documentclass[examenvragen.tex]{subfiles}

\begin{document}
\section{Robotarm}
\subsection{Opgave}
Hoe kan men bij splinecurven punten laten interpoleren door het samennemen van controlepunten of knooppunten?

\subsection{Antwoord}
Stel dat de splinecurve $s$ die we bespreken van graad $n$ is, dan heeft ze dus $p+n$ `de Boor punten' en $p+2n+1$ knooppunten.

Als we $n$ (of $n+1$) knooppunten kunnen laten samenvallen interpoleert de curve in een controlepunt. Beschouw de curve op een punt met $u \in [u_l,u_{l+1})$ waarbij we de knooppunten $u_{l-n},u_{l-n+1},\cdots, u_{l-1}$ laten samenvallen.
\[
s(u)
= \sum_{i=-n}^{p-1}d_iN_{i}^{n}(u)
= \sum_{i=l-n}^{l}d_iN_{i}^{n}(u)
= d_l
\]
Als we $n$ (of $n+1$) controlepunten $d_{l-n},d_{l-n+1},\cdots, d_{l-1}$ laten samenvallen interpoleert de curve in dat samenvallend controlepunt.
\[
s(u)
= \sum_{i=-n}^{p-1}d_iN_{i}^{n}(u)
= \sum_{i=l-n}^{l}d_iN_{i}^{n}(u)
= d_i\sum_{i=l-n}^{l}N_{i}^{n}(u)
= d_{l-1}\left(\sum_{i=l-n}^{l-1}N_{i}^{n}(u)\right) + d_{l}N_{l}^{n}(u)
\]
In $u_l$ is deze uitdrukking gelijk aan $d_{l-1}$.

\end{document}
