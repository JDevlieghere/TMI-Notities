\documentclass[computergesteund_ontwerp_van_curven_en_oppervlakken.tex]{subfiles}
\begin{document}

\section*{Voorwoord}
Deze tekst is tot stand gekomen toen duidelijk werd dat de tekst van Professor Roose \cite{tmi} voor mij persoonlijk te hoog gegrepen was. Ik heb nood aan een uitgebreidere uitleg en bewijzen van eigenschappen. De eigenschappen zijn me meestal niet zo evident. Deze tekst steunt daarom op de volgende punten.
\begin{itemize}
\item \textbf{Theorie in plaats van praktijk}\\
De praktische kant van de zaak kan sommigen helpen om de theorie beter te begrijpen. Voor mij persoonlijk maakt het het vaak alleen moeilijker om de theorie te begrijpen. Het wordt soms moeilijk om een voorbeeld van theorie te onderscheiden zodat abstractie nagenoeg onmogelijk wordt. Bovendien maken voorbeelden tussen theorie de cursus vaak onnodig dikker.

\item \textbf{Bewijzen}\\
Omdat eigenschappen me vaak helemaal niet meteen evident zijn heb ik nood aan bewijzen. Niet alleen ben ik er dan zeker van dat de eigenschap geldt, maar het verloop van het bewijs maakt me vaak duidelijk \emph{waarom} de eigenschap geldt.

\item \textbf{Extra gedetailleerde uitleg}\\
Zelfs wanneer het bewijs van een eigenschap gegeven wordt in de cursus, gaat het voor mij persoonlijk vaak te snel. Er worden stappen over geslagen die me niet evident zijn. Dat, of ze worden niet uitgelegd. Ik heb het nodig om minder stappen over te slaan, en de stappen meer uit te leggen. Ik heb daarom ook geprobeerd de gedachtegang zo veel mogelijk uit te schrijven in de bewijzen in deze tekst.

\end{itemize}

\end{document}
