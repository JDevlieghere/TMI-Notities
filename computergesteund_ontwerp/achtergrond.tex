\documentclass[computergesteund_ontwerp_van_curven_en_oppervlakken.tex]{subfiles}
\begin{document}

\chapter{Achtergond}
\section{De affiene ruimte}
\begin{de}
Een \textbf{affiene ruimte} is een tupel $(A,V)$ van een verzameling $A$ en een vectorruimte $V$. Op $A$ is een optelling met $V$ gedefinie\"erd.
\[
l: V \times A \rightarrow A: \vec{v}+a
\]
Een affiene ruimte heeft de volgende eigenschappen.
\begin{enumerate}
\item Linker Identiteit
\[
\forall a \in A: \vec{0} + a = a
\]
\item Associativiteit
\[
\forall v, w \in V, \forall a \in A,\; \vec{v} + (\vec{w} + a) = (\vec{v} + \vec{w}) + a
\]
\item Uniciteit
\[
\forall a \in A,\; V \to A\colon v \mapsto v + a \text{ is een bijectie}\]
\end{enumerate}
Vanwege de uniciteitseigenschap is de aftrekking van twee elementen $a_1$ en $a_2$ gedefini\"eerd (als een vector $v$). In de rest van deze notities zal de verzameling $A$ een verzameling punten zijn en $V$ een verzameling verschilvectoren.
\end{de}


\section{Combinaties van punten}
Punten kunnen niet bij elkaar opgeteld worden, en niet gescaleerd worden omdat er geen oorsprong is en geen assenstelsel in een affiene ruimte.
Het verschil van twee punten is wel gedefinieerd, namelijk als een vector. Bovendien is de som van een punt en een vector ook gedefinieerd.
\subsection{Affiene combinatie}
\label{affiene_combinatie}
\begin{de}
Een \textbf{affiene combinatie} $c$ van punten van een affiene ruimte $(A,V)$ is een lineaire combinatie van punten $p_i$ waarvan de gewichten $a_i$ sommeren tot $1$.
\[
c = \sum_{i}a_ip_i \text{ met } \sum_{i}a_i = 1
\]
\end{de}
\subsubsection{Een cruciaal inzicht}
Zij $p_1,...p_n$ $n$ punten van een affiene ruimte $(A,V)$ en $\alpha_i$, scalars die sommeren tot $1$.
\[
\sum_{i}a_i = 1
\]
\begin{ei}
De affiene combinatie van de $p_i$ kan gezien worden als de optelling van $n-1$ vectoren bij een punt.
Er bestaan $v_i$ zodat volgende bewering geldt.
\[
x = \sum_{i=1}^n\alpha_ip_i = p_0 + \sum_{i=2}^n\vec{v_i}
\]

\begin{proof}
We zullen dit bewijzen door de $v_i$ te construeren.
\[
\sum_{i=1}^n\alpha_ip_i
= \left(1-\sum_{i=2}^n\alpha_i\right)p_1 + \sum_{i=2}^n\alpha_ip_i
= p_1 - \sum_{i=2}^n\alpha_ip_1 + \sum_{i=2}^n\alpha_ip_i
= p_1 + \sum_{i=2}^n\alpha_i(p_i-p_1)
\]
We weten dat de $p_i-p_1$ vectoren zijn. Noem ze $v_i$.
\end{proof}
\end{ei}

Het is nu eenvoudig te zien dat een lineaire combinatie van punten niet bepaald is.
Opdat de combinatie te herschrijven zou vallen als de som van een punt en een aantal vectoren moeten de gewichten van de lineaire combinatie sommeren tot $1$. 

\subsection{Convexe combinatie}
\label{convexe_combinatie}
\begin{de}
Een \textbf{convexe combinatie} $c$ van punten $p_i$ van een affiene ruimte is een affiene combinatie van die punten met enkel positieve gewichten. Of nog, een convexe combinatie van punten is een lineaire combinatie van die punten met positieve gewichten die sommeren tot $1$.
\[
c = \sum_{i}a_ip_i \text{ met } \sum_{i}a_i = 1 \text{ en } \forall i: a_i \ge 0
\]
\end{de}

\subsection{Affiene transformatie}
\label{affiene_transformatie}
\begin{de}
Een \textbf{affiene transformatie} $T: (A,V) \rightarrow (A,V) $ is een transformatie die elke affiene combinatie van punten afbeeldt op diezelfde affiene combinatie van de beelden van die punten.\\
In symbolen: Zij $x$ een affiene combinatie van de punten $p_i$.
\[
x = \sum_{i=1}^{n}\alpha_ip_i
\]
Dan geldt de volgende gelijkheid.
\[
T(x) = \sum_{i=1}^{n}\alpha_iT(p_i)
\]
\end{de}

\section{Lijnstuk en rechte door twee punten}
\begin{de}
\label{rechte}
De rechte $l$ door twee punten $p_1$ en $p_2$ is de verzameling van alle affiene combinaties van $p_1$ en $p_2$.
\[
l = \{\ tp_1 + (1-t)p_2\ |\ t\ \in \mathbb{R}\ \}
\]
\end{de}
\begin{de}
\label{lijnstuk}
Het lijnstuk $l$ door twee punten $p_1$ en $p_2$ is de verzameling van alle convexe combinaties van $p_1$ en $p_2$.
\[
l = \{\ tp_1 + (1-t)p_2\ |\ t\ \in \mathbb{R}\wedge\ ;0\le t\le 1\ \}
\]
\end{de}

\section{Convex omhullende}
\label{convex_omhullende}
\begin{de}
Een \textbf{verzameling} $M$ is \textbf{convex} als voor elk lijnstuk dat twee punten van $M$ verbindt, alle punten van het lijnstuk tot $M$ behoren.
\[
\forall p_1,p_2 \in M,\ \forall t,\ 0\le t\le 1\ \Rightarrow tp_1+(1-t)p_2 \in M
\]
\end{de}
\begin{de}
De \textbf{convex omhullende} $CH(M)$ van een verzameling $M$ is de kleinste convexe verzameling die $M$ omvat.
\end{de}
\begin{st}
Een convexe combinatie van punten behoort tot de convex omhullende van deze punten.
\begin{proof}
Bewijs door inductie op het aantal punten ($n$).\\
Voor $n=1$ is de convex omhullende enkel dat punt en elke convexe combinatie van een punt is dat punt.\\
Veronderstel dat de bewering geldt voor een bepaalde $n=k$. We bewijzen nu dat daaruit volgt dat de bewering geldt voor $n=k+1$.\\
Zij $x$ de convexe combinatie van de punten $p_1,...,p_k,p_{k+1}$.
\[
x = \sum_{i=1}^{k+1}\alpha_ip_i
\]
Als $\alpha_{k+1} = 1$ geldt dan is de bewering triviaal. In wat volgt gaan we ervan uit dat $\alpha_{k+1} \neq 1$ geldt.
We weten nu dat volgende bewering geldt, omdat het om een convexe combinatie gaat.
\[
\sum_{i=1}^{k}\alpha_i = S \Rightarrow 0 \le S \le 1
\]
Beschouw nu een punt $y$ dat als volgt gedefinieerd is met $\beta_i = \frac{\alpha_i}{S}$.
\[
y = \sum_{i=1}^k\beta_{i}p_{i}
\]
We sommeren nu de $\beta_i$ om te zien dat $y$ ook een convexe combinatie is.
\[
\sum_{i=1}^k\beta_{i} = \sum_{i=1}^k \frac{\alpha_i}{S} = \frac{1}{S}\sum_{i=1}^k \alpha_i = \frac{\sum_{i=1}^{k}\alpha_i}{\sum_{i=1}^{k}\alpha_i} = 1
\]
Volgens de inductiehypothese ligt $y$ in de convex omhullende van $p_1,...,p_k$. $y$ ligt dus zeker in de convex omhullende van $p_1,...,p_k,p_{k+1}$. Nu geldt het volgende over $x$ met $1-S = \alpha_{k+1}$.
\[
x = Sy + (1-S)p_{k+1}
\]
Dit is opnieuw een convexe combinatie, van twee punten binnen de convex omhullende. $x$ behoort dus tot de convex omhullende van $p_1,...,p_k,p_{k+1}$. 
\end{proof}
\end{st}
\begin{st}
De convex omhullende $CH(M)$ van een verzameling $M$ is de verzameling $N$ van alle convexe combinaties van punten in $M$.
\[
CH(M) = \{ tp_1 + (1-t)p_2\ |\ p_1,p_2 \in M, 0\le t \le 1\}
\]
\begin{proof} Bewijs in delen.
\begin{itemize}
\item $N$ is een convexe verzameling\\ Inderdaad. Alle lijnstukken (verzamelingen convexe combinaties van twee punten)  die twee punten van $M$ verbinden behoren tot $M$.
\item Er bestaat geen kleinere convexe verzameling $N'$, die $M$ omvat, dan $N$:\\
Stel dat $N'$ bestaat, dan bestaat er minstens \'e\'en punt $p' \in N$ dat niet in $N'$ zit. $N'$ zou echter niet meer convex zijn, want het punt $p'$ is een convexe combinatie van elementen in $M$.
\end{itemize}
\end{proof}
\end{st}

\begin{gev}
De convex omhullende van twee punten $p_1$ en $p_2$ is het lijnstuk tussen $p_1$ en $p_2$.
\[
CH(\{p_1, p_2)\} = \{\ tp_1 + (1-t)p_2\ |\ t\ \in \mathbb{R}\wedge\ ;0\le t\le 1\ \} = N
\]
\end{gev}



\section{Interpolerende veeltermen}
\subsection{Lagrange veelterm}
\begin{de}
Bij $n+1$ punten $p_i$ horen de volgende $n$ \textbf{Lagrange veeltermen} $L_i$.
\[
L_{i}^{n}(x) = \prod_{j=0, j \neq i}^{n} \frac{x-x_{j}}{x_{i}-x_{j}}
\]
\end{de}

\begin{ei}
\label{cd_one}
Voor de $i$-de Lagrange veelterm geldt dat deze $1$ is in $x_i$
\[
\forall i:\ L_{i}^{n}(x_i) = 1
\]
\begin{proof}

\[
L_{i}^{n}(x_i)
= \prod_{j=0, j \neq i}^{n} \frac{x_i-x_{j}}{x_{i}-x_{j}}
= \prod_{j=0, j \neq i}^{n} 1
= 1
\]
\end{proof}
\end{ei}

\begin{ei}
\label{cd_two}
Voor de $i$-de Lagrange veelterm geldt dat deze $0$ is in $x_k$ wanneer $k\neq i$ geldt.
\[
\forall i,k:i\neq k \Rightarrow L_i^{n}(x_k) = 0
\]
\begin{proof}
\[
L_{i}^{n}(x_k)
= \prod_{j=0, j \neq i}^{n} \frac{x_k-x_{j}}{x_{i}-x_{j}}
= \frac{x_k-x_{1}}{x_{i}-x_{1}}\cdot \ldots \cdot 0 \cdot \ldots \cdot \frac{x_k-x_{n}}{x_{i}-x_{n}}
= 0
\]
\end{proof}
\end{ei}

\begin{gev}
\label{lagrange_veeltermen_sommeren_tot_een}
We kunnen de interpolerende veelterm $p(t)$ van $n+1$ punten $p_1$ als volgt berekenen. Volgens eigenschappen \ref{cd_one} en \ref{cd_two} interpoleert deze veelterm de punten $p_i$.
\[
p(t) = \sum_{i=0}^{n}L_{i}^{n}(t)p_i
\]
\end{gev}

\begin{ei}
De som van alle $n+1$ lagrange veeltermen is steeds $1$. (Gevolg \ref{lagrange_veeltermen_sommeren_tot_een} is daarom zinvol.)
\[
\forall t: \sum_{i=0}^{n}L_i^{n}(t) = \sum_{i=0}^{n}\prod_{j=0, j \neq i}^{n} \frac{t-x_{j}}{x_{i}-x_{j}} = 1
\]
\begin{proof}
Stel de interpolerende veelterm $f(x)$ van graad $n$ van de constante functie $y=1$ op.
We weten nu dat elke $f(x_i)$ gelijk is aan $1$ en dat $y_n$ $y$ exact benadert.
Bijgevolg geldt het volgende.
\[
\sum_{i=0}^{n}\prod_{j=0,j\neq i}^{n}\frac{x-x_j}{x_i-x_j} = 1
\]
\end{proof}
\end{ei}



\end{document}
