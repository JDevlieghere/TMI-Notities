\documentclass[examenvragen.tex]{subfiles}

\begin{document}
\section{Diagonaal}
\subsection{Opgave}
Gegeven een verzameling van $N$ punten. Ontwerp een algoritme om te berekenen welk paar punten het verst uit elkaar liggen.

\subsection{Antwoord}
Een naief algoritme werkt natuurlijk in $O(N^2)$ tijd. We kunnen de complexiteit echter verlagen tot $O(N\log(N))$ als volgt.
We berekenen de convex omhullende van de verzameling punten in $O(N\log(N))$.
Daarna berekenen we alle tegenvoetersparen in $O(N)$ tijd.
Tenslotte overlopen we de tegenvoetersparen om het verste puntenpaar te vinden in $O(N)$. In totaal heeft dit algoritme $O(N\log(N))$ tijd nodig.

Dit algoritme is correct omdat het verste puntenpaar steeds op de convex omhullende ligt. We maken bovendien gebruik van het feit dat we alle tegenvoetersparen kunnen vinden in $O(N)$ tijd.

\end{document}
