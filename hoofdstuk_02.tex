\documentclass[tmi_notities.tex]{subfiles}
\begin{document}
\chapter{B\'ezier-curven}
\section{Definities}
$i$ Bernstein-veelterm van graad $n$:
\[
B_{i}^{n} = \binom{n}{i}(1-t)^{n-i}t^{i}
\]
$B_{i}^{n} = 0$ als $i < 0$ of $i > n$.\\\\
B\'ezier curve met $n+1$ controlepunten:
\[
\vec{x}(t) = \sum_{i=0}^n\vec{b_{i}}B_{i}^{n}(t)
\]

\section{Eigenschappen van Bernstein-veeltermen}
\begin{enumerate}
\item
\[
\sum_{i=0}^nB_{i}^{n}(t) = 1
\]
\begin{proof}
\[
\sum_{i=0}^nB_{i}^{n}(t)
= \sum_{i=0}^{n}\binom{n}{i}(1-t)^{n-i}t^i
= ((1-t)+t)^n
= 1
\]
\end{proof}

\item
Voor $0 \le t \le 1$:
\[
B_{i}^{n}(t) \ge 0
\]
\begin{proof}
\[
\binom{n}{i}(1-t)^{n-i}t^{i}
\]
$\binom{n}{i}$ is positief, $(1-t)$ is positief want $0 \le t \le 1$ en $t$ is ook positief.
\end{proof}

\item Zij $n>0$.
\begin{itemize}
\item $B_{i}^{n}(0) = B_{i}^{n}(1) = 0$ als $i\neq$, $i\neq n$.
\item $B_{0}^{n}(0) = 1$
\item $B_{0}^{n}(1) = 0$
\item $B_{n}^{n}(0) = 0$
\item $B_{n}^{n}(1) = 1$
\end{itemize}
\begin{proof}
Vul eenvoudigweg alles in.
\begin{itemize}
\item $B_{i}^{n}(0) = B_{i}^{n}\binom{n}{i}1^{n-i}0^{i} = 0$ en $B_{i}^{n}(1) = B_{i}^{n}\binom{n}{i}0^{n-i}1^{i} = 0$
\item $B_{0}^{n}(0) = 1^{n} = 1$
\item $B_{0}^{n}(1) = 0^{n} = 0$

\item $B_{n}^{n}(0) = 0^{n} = 0$
\item $B_{n}^{n}(1) = 1^{n} = 1$
\end{itemize}
\end{proof}

\item Symmetrie
\[
B_{i}^{n}(t) = B_{n-i}^{n}(1-t)
\]
\begin{proof}
\[
\binom{n}{n-i}(1-1+t)^{n-n+i}(1-t)^{n-i}
= \binom{n}{i}t^{i}(1-t)^{n-i}
= \binom{n}{i}(1-t)^{n-i}t^{i} 
\]
\end{proof}

\item Afgeleide
\[
\frac{d}{dt}B_{i}^{n}(t) = n(B_{i-1}^{n-1}(t) - B_{i}^{n-1}(t))
\]
\begin{proof}
\[
\frac{d}{dt}B_{i}^{n}(t) = \frac{d}{dt}\binom{n}{i}(1-t)^{n-i}t^{i}
\]
\[
\binom{n}{i}\frac{d}{dt}\left((1-t)^{n-i}t^{i}\right)
= \binom{n}{i}
\left(
t^{i}\frac{d}{dt}(1-t)^{n-i}
+ (1-t)^{n-i}\frac{d}{dt}t^{i}
\right)
\]
\[
\binom{n}{i}\frac{d}{dt}\left((1-t)^{n-i}t^{i}\right)
= \binom{n}{i}
\left(
(1-t)^{n-i}it^{i-1}
-t^{i}(n-i)(1-t)^{n-i-1}
\right)
\]
\[
= 
\left(
\frac{n!}{i!(n-i)!}
(1-t)^{n-i}it^{i-1}
-
\frac{n!}{i!(n-i)!}
t^{i}(n-i)(1-t)^{n-i-1}
\right)
\]
\[
= 
n
\left(
\frac{(n-1)!}{(i-1)!(n-i)!}
(1-t)^{n-i}t^{i-1}
-
\frac{(n-1)!}{i!(n-i-1)!}
t^{i}(1-t)^{n-i-1}
\right)
\]
\[
 = n(B_{i-1}^{n-1}(t) - B_{i}^{n-1}(t))
\]
\end{proof}

\item
\[
\frac{d}{dt}B_{i}^{n}(t) = 0 \Leftrightarrow t = \frac{i}{n} \text{ voor } n > 0 \text{ en } 0 < i < n
\]
\begin{proof}
\[
\frac{d}{dt}B_{i}^{n}(t) = n(B_{i-1}^{n-1}(t) - B_{i}^{n-1}(t)) = 0
\]
\[
nB_{i-1}^{n-1}(t) = nB_{i}^{n-1}(t)
\]
\[
n \frac{(n-1)!}{(i-1)!(n-i)!}
(1-t)^{n-i}t^{i-1}
=
n \frac{(n-1)!}{i!(n-i-1)!}
(1-t)^{n-i-1}t^{i}
\]
\[
\frac{(1-t)}{n-i}
=
\frac{t}{i}
\]
\[
t = \frac{i}{n}
\]
\end{proof}

\item Recursiebetrekking.
\[
B_{i}^{n}(t) = (1-t)B^{n-1}_{i}(t) + tB^{n-1}_{i-1}(t)
\]
\begin{proof}
\[
(1-t)B^{n-1}_{i}(t) + tB^{n-1}_{i-1}(t)
\]
\[
= 
(1-t)\binom{n-1}{i}
(1-t)^{n-i-1}t^{i}
+
t\binom{n-1}{i-1}
(1-t)^{n-i}t^{i-1} 
\]
\[
= \left(\binom{n-1}{i}+\binom{n-1}{i-1}\right)
(1-t)^{n-i}t^{i}
\]
\[
=
\binom{n}{i}
(1-t)^{n-i}t^{i}
= B_{i}^{n}(t)
\]
\end{proof}
\end{enumerate}

\section{Eigenschappen van B\'ezier-curven}
B\'ezier curve met $n+1$ controlepunten:
\[
\vec{x}(t) = \sum_{i=0}^n\vec{b_{i}}B_{i}^{n}(t)
\]
\begin{enumerate}
\item
Een B\'ezier-curve is een veeltermcurve van graad $n$.
\begin{proof}
\[
\sum_{i=0}^n\vec{b_{i}}B_{i}^{n}(t) = \sum_{i=0}^n\vec{b}_i\frac{n!}{i!(n-i)!}(1-t)^{n-i}t^{i}
\]
De hoogste graad van $t$ komt voor wanneer $i$ nul is. Dan is die graad $n$.
\end{proof}

\item
$\vec{x}(t)$ is een affiene combinatie van de controlepunten $\vec{b}_i$. $\vec{x}(t)$ is dus een punt dat afhangt van de parameter $t$.
\begin{proof}
De som van alle $B_{i}^{n}(t)$ is $1$. Zie eigenschap $1$ van Bernstein-veeltermen. Dit geldt voor elke $t$.
\end{proof}

\item 
Het verband tussen de B\'ezier curve en de B\'ezier punten is invariant onder affiene transformaties.
\begin{proof}
Zij $L$ een affiene transformatie.
\[
L(\vec{x}(t))
= L\left( \sum_{i=0}^n\vec{b_{i}}B_{i}^{n}(t)\right)
= \sum_{i=0}^nL(\vec{b_{i}})B_{i}^{n}(t)
\]
\end{proof}

\item Elk punt op een B\'ezier curve light binnen de convex omhullende van de controlepunten.
\begin{proof}
Elk punt van de Bezier-curve is een convexe combinatie van de controle punten want elke Bernstein-veelterm is positief (zie eigenschap $2$ van Berstein-veeltermen.). Omdat elk punt op de curve een convexe combinatie is, ligt elk punt binnen de convex omhullende van de controlepunten.
\end{proof}

\item De B\'ezier curve interpoleert in het eerste en laatste controlepunt.
\begin{proof}
In $t=0$ interpoleert de B\'ezier curve in het eerste controlepunt. In $t=1$ interpoleert de B\'ezier curve in het laatste controlepunt.
\begin{itemize}
\item
\[
\vec{x}(0) = \sum_{i=0}^n\vec{b_{i}}B_{i}^{n}(0) = B_{0}^n\vec{b}_0 = \vec{b}_0
\]

\item
\[
\vec{x}(1) = \sum_{i=0}^n\vec{b_{i}}B_{i}^{n}(1) = B_{n}^n\vec{b}_n = \vec{b}_n
\]
\end{itemize}
\end{proof}

\item Variatie-verminderingseigenschap\\
Het aantal snijpunten van een willekeurige rechte (in 2D) of vlak (in 3D) met de B\'ezier-curve is kleiner dan of gelijk aan het aantal snijpunten van die rechte (of vlak) met de B\'ezier-veelhoek.
\begin{proof}
TODO
\end{proof}

\end{enumerate}

\section{Evaluatie van een punt op een B\'ezier curve (de Castejau)}
Zij $\vec{b}_{i}^{[0]} = \vec{b}_{i}$ en definieer $\vec{b}_{i}^{[k]}$ recursief als volgt.
\[
\vec{b}_{i}^{[k]}
=
(1-t)\vec{b}_{i-1}^{[k-1]} + t\vec{b}_{i}^{[k-1]}
\]
Nu geldt volgende bewering. 
\[
\vec{b}_{n}^{[n]} = \sum_{i=0}^{n}\vec{b}_iB_{i}^{n}(t) = \vec{x}(t)
\]
\begin{proof}
\[
\vec{x}(t) = \sum_{i=0}^{n}\vec{b}_iB_{i}^{n}(t)
\]
Gebruik de recursiebetrekking voor bernstein-veeltermen.
\[
= \sum_{i=0}^{n}\vec{b}_i
\left( 
(1-t)B^{n-1}_{i}(t) + tB^{n-1}_{i-1}(t)
\right)
= 
\sum_{i=0}^{n}\vec{b}_i
(1-t)B^{n-1}_{i}(t) + 
\sum_{i=0}^{n}\vec{b}_it
B^{n-1}_{i-1}(t)
\]
$B_{i}^{n} = 0$ als $i < 0$ of $i > n$.
\[
= 
\sum_{i=0}^{n-1}\vec{b}_i
(1-t)B^{n-1}_{i}(t) + 
\sum_{i=1}^{n}\vec{b}_it
B^{n-1}_{i-1}(t)
\]
Merk op dat alle $\vec{b}_i$ gelijk zijn aan $\vec{b}_i^{[0]}$. We proberen ze nu zo te schrijven dat we er $\vec{b}_i^{[1]}$ van kunnen maken. Vervang de $i$ in de eerste term door $j=i+1$ en hernoem daarna $j$ naar $i$. In feite schuiven we de sommatie-index $1$ \'e\'enheid op.
\[
= 
\sum_{i=1}^{n}\vec{b}_{i-1}
(1-t)B^{n-1}_{i-1}(t) + 
\sum_{i=1}^{n}\vec{b}_it
B^{n-1}_{i-1}(t)
\]
Nu zien we dat de sommaties dezelfde zijn. We kunnen deze dus buiten brengen omdat de optelling commutatief is.
\[
= 
\sum_{i=1}^{n}
\left(
\vec{b}_{i-1}(1-t)B^{n-1}_{i-1}(t)
+ 
\vec{b}_itB^{n-1}_{i-1}(t)
\right)
\]
\[
= 
\sum_{i=1}^{n}
B^{n-1}_{i-1}(t)
\left(
(1-t)\vec{b}_{i-1}
+ 
t\vec{b}_i
\right)
\]
Kijk nu naar de zojuist gedefini\"eerde recursiebetrekking. We kunnen $(1-t)\vec{b}_{i-1} + t\vec{b}_i = (1-t)\vec{b}_{i-1}^{[0]} + t\vec{b}_i^{[0]}$ vervangen door $\vec{b}_{i}^{[1]}$
\[
= 
\sum_{i=1}^{n}
B^{n-1}_{i-1}(t)
\vec{b}_{i}^{[1]}
\]
Als we dit proces nog eens herhalen krijgen we de volgende uitdrukking.
\[
= 
\sum_{i=2}^{n}
B^{n-2}_{i-2}(t)
\vec{b}_{i}^{[2]}
\]
In de $j$-de iteratie ziet het nieuwe rechterlid er als volgt uit.
\[
= 
\sum_{i=j}^{n}
B^{n-j}_{i-j}(t)
\vec{b}_{i}^{[j]}
\]
Dit proces kunnen we $n$ (dit eindig en zelfs lineair) keer uitvoeren tot we volgende uitdrukking bekomen.
\[
\sum_{i=n}^{n}
B^{0}_{0}(t)
\vec{b}_{i}^{[n]}
= \vec{b}_{n}^{[n]}
\]
\end{proof}
Deze bewering geeft een algoritme om een punt op de B\'ezier-curve recursief te berekenen. De index in superscript tussen vierkante haakjes heeft betrekking tot de iteratie waarin dit punt zich bevindt.

\section{Afgeleide van een B\'ezier-curve}


\end{document}

